% !TEX root = ./main.tex
\section{Discussion}\label{sec:discussion}
Intro to section

\et{I suggest to split this section in two parts: one that is related to the core (schema, data model, queries, semantics), and on that is related to normalization/simplified semantics. And to move the first to just after section 3 (either as a 3.5 or as a separate section 4 if it's length justifies it)}

\subsection{\gql Schema}

\et{I don't get what is the point of this paragraph, and why it matters -- what's the takeaway message? how does it relate to what we did?}
We believe that the \spec's definition of the schema can be slightly confusing and ambiguous. The schema is described as being ``\textit{defined in terms of the types and directives it supports as well as the root operation types for each kind of operation}''\footnote{https://graphql.github.io/graphql-spec/June2018/\#sec-Schema}. However, the \spec also defines a structure called \texttt{schema}\footnote{https://graphql.github.io/graphql-spec/June2018/\#SchemaDefinition} that only contains the root operation types (query, mutation and subscription), meanwhile the type definitions and directives are defined separately. We think this introduces the first ambiguity to the definition. Now, to actually capture the notion described above, it is necessary to define what is called a \textit{Document}\footnote{https://graphql.github.io/graphql-spec/June2018/\#Document}, consisting of a list of definitions and queries. Among these definitions one can include the \texttt{schema} structure, as well as the type definitions and directives (by repeatedly using the \textit{Type System Definition}\footnote{https://graphql.github.io/graphql-spec/June2018/\#TypeSystemDefinition} rule). This effectively permits building the schema as expected, however we think it is unintuitive to allow mixing definitions with queries. Also, the naming of the different structures seems to introduce confusion and inconsistency.

 
% the \spec defines each of the three elements (types, directives and root operation types) as separate entities. The first candidate is the \textit{Type System} \footnote{https://graphql.github.io/graphql-spec/June2018/\#TypeSystemDefinition} but it is a disjunction of the three elements: types, directives and root operation types. 


% Secondly, the \spec defines the \texttt{schema}\footnote{https://graphql.github.io/graphql-spec/June2018/#SchemaDefinition} structure that only contains the root operation types (query, mutation and subscription). The type definitions and directives are defined separately. Even though it is not a major difference, it lends itself to possible confusion when referring to the schema of a \gql service. In addition, the previously quoted definition does


%The previously quoted definition actually matches the \textit{Type System} structure\footnote{https://graphql.github.io/graphql-spec/June2018/\#TypeSystemDefinition}. Our formalization follows the latter but rename it to schema to also match the quoted description.

\et{contrast with what you did, and the tradeoffs (what's convenient, what's not)}
Regarding \HP's consistency property, they embed many properties in their structures, such as uniqueness of types given by using sets. They include an additional check on objects implementing interfaces, where they validate that fields are properly implemented. The definition given is not complete due to missing validation on arguments, but a corrected version is included in \cite{olafschema}.

\subsection{Data model}

As described in Section~\ref{subsec:graph}, \gql is agnostic to the technology used and the underlying data model. We follow \HP and instantiate the semantics to a graph setting, allowing to reason about it\et{see my comment in the intro}. We describe here severe limitations \et{why is the list nesting situation "severe"?? what backs up this characterization?} that this data model introduces on the possible results generated and the open questions regarding how to properly model certain aspects of \gql schemas \et{vague}. This model is exploited by \HP and \cite{olafschema} but neither discuss nor mention the limitations. \et{you only discuss one, without really explaining why it matters, or to what extent. Does it break all established results? etc.}

The main issue with this graph model is that there is no proper accounting of list types containing other list types (with any nesting depth). When it comes to list types it is not clear what they represent in a graph. Let us illustrate this with an example.
%The different features that compose a \gql schema can be represented in a graph somehow. For instance, a field is either a property or the label of an edge, while its return type can be associated to a target node in an edge. However, when it comes to list types it is not clear what they represent in a graph. Let us illustrate this with an example.


% Our definition is in essence the same as in \HP but differs greatly in implementation. \HP defines a \gql graph in a more ``centralized'' manner. For instance, nodes and field names are defined by sets. Node types are defined by a single function which receives a node identifier and gives its type. Properties are also defined by a single function which receives a node identifier and a field name with arguments. Contrarily, our approach attempts to recreate the structures individually. For instance, a node contains all the information pertaining to itself; its type and its properties. We believe this is a more natural approach to defining the graph from an engineering point of view.

A service may declare the field \texttt{friends:[Human]} in a given type, representing the list of friends.
In a graph this can be pictured as having a node with multiple outgoing edges labeled \texttt{friends}, reaching other nodes of type \texttt{Human}. It is possible to then extend the service by including a new field \texttt{friendsByName:[[Human]]}, in which one can request a list of friends but grouped by their names. At the moment neither our implementation, \HP nor \cite{olafschema} properly handle this situation. The open question is what does this represent in the graph? These should be outgoing edges similarly to the previous case but, what should the target nodes be? Should these be intermediate blank nodes? Is every edge labeled or only the last one that reaches a node with type \texttt{Human}? What happens if we increase the nesting? Since the information is ultimately collected from ``concrete'' nodes, should the graph be kept the same but introduce \textit{formatter} functions to match the schema? How does this differ from the \gql resolvers?

These questions and more \et{what are the more? instead of spending half a page on nested lists, tell us what are the other limitations and what their possible impact is -- ie. like a typical "threats to validity" section} are not addressed nor discussed in \HP and it is actually more restrictive than expected\et{expected by whom?}, by not allowing nested lists for scalar values (in nodes's properties)\et{in the previous paragraph, you put all three approaches in the same bag, now it seems \HP does worse}. Meanwhile, our approach and the one used in \cite{olafschema} allow any list type at the property level but simply ignore any possible nesting when the list type refers to neighboring nodes (composite types), as in the example above.\et{for instance, a typical question would be: "how prevalent are those situations?" if you don't know the answer to this question, how can you evaluate if it's severe or anecdotical?}
In the case of \cite{olafschema}, they do not address nor discuss these questions\et{you said that already}. This choice of modeling has some consequences\et{what are they?} when defining the semantics of \gql queries, because the possible results generated are restricted to a smaller subset\et{the limitation seemed to be about the data model, not the queries or their results}. It is not clear what the proper way is to handle this issue\et{still talking about nested lists?} but more is explored in Section~\ref{subsec:semantics}\et{watch out: 3.4 is now *before* this text}. 
We also address the \spec's semantics and how this is managed.\et{?}

\subsection{Queries}


Both the \spec and our formalization differ from \HP when defining queries. The main difference is that \HP include an additional rule for lists of queries. Their grammar includes a production rule for lists of queries, which is defined at the same level of the other rules. The main issue we found with this approach is that it allows building arbitrary trees instead of just a list of queries. These trees can be flattened to recover the list structure but increasing the effort when defining functions and reasoning over queries. We believe this is assumed by \HP but not explicitly mentioned otherwise.


On a different note, we mentioned in Section~\ref{subsec:query} that we split a validation rule into two separate predicates. The reason behind this is that we noticed that the \spec's definition includes redundant recursive calls which may result in increased computational time. By splitting the definition in two parts, we expect to optimize the algorithm and also facilitate reasoning about them. At the time of writing this paper, a new algorithm was proposed by a team at XING\footnote{https://www.xing.com/} that also addresses this very same issue and is described in~\cite{xingalg}. They follow a different approach to resolving it, using sets, and provide a much more elaborate analysis of execution times than us. Comparing both approaches and analyzing execution times could be an interesting venue to explore.


Finally, we also noticed that the previous rule is too conservative and may consider valid queries as invalid. This occurs because the \spec allows defining fragments that are never evaluated. The issue is that the validation rule can then consider the subqueries in these fragments as invalid, even though they are never evaluated, rendering the whole query invalid\footnote{An example query can be seen in the following link: https://tinyurl.com/y3hz5vgv.}. We attempt to remove this conservativeness in the predicate that checks the merger of fields but we have not provided proofs that it does. For the predicate regarding unambiguous results, it is equally conservative as the \spec. We believe it should not be hard to modify the definition, however we decided against it at the moment, to keep it simple and facilitate reasoning over the predicate, as well as preserving some similarity to the \spec\td{Although it may not be entirely similar from the start...}.


\subsection{Semantics}

%We finish this section by addressing two major aspects about our formalization; completeness and errors.

The first one was briefly mentioned in Section~\ref{subsec:graph}, when discussing the limitations and open questions regarding the graph model. These translate in the fact that we currently do not produce list results with nested lists of objects. For instance, the field \texttt{friendsByName:[[Human]]} is treated as if it were defined as \texttt{friendsByName:[Human]} and the results match the latter format. Otherwise, there is no restriction in the case of nested lists for scalar values. In \HP, there is no possibility to produce nested lists for either scalar or object values\footnote{The grammar itself does not permit it.} and there is no mention of this restriction.

Regarding error handling, we currently do not implement it. Errors may have two main sources; validation errors and execution errors.\td{Not sure how to write this}

\subsection{Normalization}

\subsubsection{Grounding}
This definition differs slightly from the one given by \HP, because they do not use information on the type in context. We take this approach because the structure of grounded queries is deterministic and it simplifies reasoning. In \HP's approach, subqueries of fields can be either fields or inline fragments, without clear conditions. For instance, the following query, inspired from the schema in Section~\ref{sec:bg}, is in ground-typed normal form but the field \texttt{goodboi} has field subqueries in one case and fragments in the other. With our approach, only the second occurrence of \texttt{goodboi} is grounded, since its return type is an interface and the subqueries are specified down to the subtypes, using fragments.

\begin{minted}[escapeinside=||, mathescape=true]{js}
        query {
            goodboi {
              name
            }
            goodboi {
              |$\ldots$| on Dog {
                name
              }
              |$\ldots$| on Pig {
                name
              }
            }
        }
\end{minted}


In this section we discuss some discoveries made regarding \HP's definitions and how we solve them. In particular, we review the non-redundancy property and the set of equivalence rules they define to normalize queries.

For the former, we notice that their definition is unsound\td{?}, in the sense that there are queries that are considered non-redundant but actually would produce redundant results. A simple example is the following valid query, that is considered as non-redundant by their definition but which, in fact, would produce two repeated values. It is a very minor slip, which happens because their definition of non-redundancy does not consider the cases of unaliased and aliased fields sharing the same response name. Our implementation addresses this by grouping fields by their response names.

\begin{minted}{js}
        query {
            name
            name:name
        }
\end{minted}


Moving onto the equivalence rules, there are three aspects we have to highlight. The first one is that rule number (2), which refers to the merging of fields with subqueries, is correct but does not preserve ordering of the queries. While this is not imposed by the \spec, it is an important feature of \gql evaluation. This is also important at the moment of defining and comparing semantic equivalence between queries.

The second aspect is about the elements they use \td{?} in their rules. In some cases they use list of queries while in some other they define it over single queries, or sometimes mix them. While this is no big issue, it was a bit confusing when trying to implement their rules in \coq.\td{Not sure how to describe this, but the thing is their rules are a bit weird. They describe rules for individual selections, but there is no... "global" rewriting. I imagine this is "simpler" to understand with their semantics, because they do not modify the queries as they evaluate them (pushing everything to the responses), but it is still weird to define it as a procedure in Coq (or even as inductive relation).}

Finally, there is an implicit notion of type in context when they describe their rules\td{and maybe a missing rule?}. This is crucial, because otherwise there are queries that cannot be normalized. For example, the following query cannot be transformed with the rules as they are.
\begin{minted}[escapeinside=||, mathescape=true]{js}
          query {
              name
              |$\ldots$| on Query {
                  name
              }
          }
\end{minted}
However, if the type in context is included, which corresponds to \texttt{Query} in this case, it is possible to do more. The queries can be wrapped in an inline fragment with type condition \texttt{Query}. Then, with a mix of other rules the normalized query can be obtained.

\td{Not sure where to mention the whole process of doing this (since it took the most of our time). Things such as:
    \begin{itemize}
        \item Trying to implement HP's rules of equivalence.
        \item Trying to work on a subset of queries with no invalid fragments.
        \item Change/Discovery of their semantics and responses.
        \item Definition of normalization in two separate functions; one for grounding and one for removing redundancy.
        \item etc.
    \end{itemize}
}
