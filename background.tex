% !TEX root = ./main.tex
\section{A Brief Introduction to \gql}\label{sec:bg}
\td{Meant to rewrite it but time's up :(}

\gql is a framework that provides a common language to define the interface to a service's data and to query it.
It provides a language to describe how the data is structured and how it can be queried. This is called the schema or type system of the service. The schema consists of types and their fields. Queries may only be performed over these types and their fields. The resolution of each field is defined by the implementors, since \gql is not tied to any particular technology.

In the rest of this section, we will introduce \gql by means of an example. We will recurrently come back to this example throughout the rest of the paper.\td{Maybe not if we don't have space lol}

\subsection*{\gql schema}

Let's picture ourselves having a database with information about dogs and pigs; the \textit{GoodBois} database. We want to define an API so our frontend developers may get the information and display it in our website. Our first step is then to describe how the data is structured and how it may be queried. This is done by means of the schema, which represents the type system of our \gql service.

Figure \ref{fig:schema_ex} depicts our type system. We define an interface for animals and two types implementing it; \texttt{Dog} and \texttt{Pig}. We know that animals have other animal friends, so we define the field \texttt{friends} whose return type is a list of other animals. We can also define enumeration types, which contain scalar values such as \texttt{GOODBOI}, and union types containing other object types. Finally, we have to define a \texttt{Query} type, which represents the entry point to our service's data. Any query that our frontend developers may do must begin by accessing this type's fields.

\begin{figure}
    \centering
    \begin{subfigure}{.5\linewidth}
    
    \begin{minted}[fontsize=\scriptsize]{js}
interface Animal {		
  name: String
  friends: [Animal]
}
type Dog implements Animal {
  name: String
  friends: [Animal]
  favoriteToy: Toy
}
type Pig implements Animal {
  name: String
  friends: [Animal]
  oink: Float
}
    \end{minted}
    \end{subfigure}%    
    \begin{subfigure}{.5\linewidth}
    \begin{minted}[fontsize=\scriptsize]{js}
type Toy {
  chewiness: Float
}
enum Goodness { 
  BESTBOI GOODBOI 
  OKBOI BADBOI 
}
union SearchResult = Dog | Pig | Toy
type Query {
  goodboi(goodness: Goodness): Animal
  search(text: String): SearchResult
}
schema {
  query: Query
}
    \end{minted}
    \end{subfigure}
    
    \caption{Example of \gql Schema.}
    \label{fig:schema_ex}
\end{figure}

This is all it takes to describe our data and how our developers can query it. It describes exactly the data they can access and which are the entry points to it. However, each field has to somehow connected to actual data. When a developer requests the field \texttt{chewiness} we have to actually get that information from somewhere.

\subsection*{Graph data model}
Since \gql does not impose a particular technology or data model, it is not simple to reason about queries and their semantics. It is the job of the service's implementor to define how  each field of a given type is resolved.

In our scenario, our data will be stored in a graph. Figure \ref{fig:graph_ex} illustrates our service's graph database. There is a root node from which every query must begin. This root node represents the \texttt{Query} type described in the schema. We also see that each node has a type, such as \texttt{Dog} or \texttt{Toy}, and properties such as their names. Each edge is also labeled with a name as defined in the schema. For instance, the edge connecting the dog named ``Casel'' is labeled \texttt{favoriteToy}, as declared in the type \texttt{Dog} in the schema.

\begin{figure}
    \centering
    \includegraphics[scale=0.23]{imgs/graph.png}
    \caption{Example of \gql graph \et{favourite $\rightarrow$ favorite}}
    \label{fig:graph_ex}
\end{figure}

Finally, now that we have defined our type system and data, our developers can proceed to query it.

\fo{The following three paragraphs were moved from \S \ref{subsec:graph} and should
be integrated into this section.} An important aspect of \gql is that it is not tied to any particular database technology and implementation. When resolving queries, \gql simply assumes the existence of \textit{resolvers}, which are internal functions defined by the user implementing a \gql service. They are not tied to any particular data model and the only requirement is that they must adhere to the schema. It is up to the user whether the resolvers access a database, return static values or even modify existing data\footnote{The \spec{} states that resolvers ``\textit{must always be side effect‐free and idempotent}'' but the definition of a resolver does not actually impose these restrictions.}. This looseness makes it hard to reason about the semantics.

With this in mind, we choose to follow \HP's approach and define the underlying data model as a graph over which queries are evaluated. With this model, the unspecified resolvers can be instantiated to concrete definitions, which allow reasoning over them. The semantics are then described as being implemented over a graph setting. Although this provides benefits when reasoning about the semantics, it also comes with some potentially severe limitations over the completeness of the possible results generated.\td{They may not actually be limitations with the model, but there are open questions on how to model some things.} We provide a more thorough explanation of these in \S\ref{sec:discussion}, as well as \HP's approach to the subject.

Informally, a \gql graph is a directed property graph, with labeled edges and typed nodes. The graph describes entities with their types and properties, as well as the relationship between them. This means that every node has properties (key-value pairs) and a type. Also, every label in an edge describes the relation between two nodes. Finally, every property or label may also contain a list of arguments (key-value pairs).


\subsection*{\gql query and response}

As previously mentioned, the queries we perform over our system must be over the types and fields defined in the schema. Every query must start by requesting information from the \texttt{Query} type. That means that, in our setting, queries must all start with the \texttt{goodness} or \texttt{search} fields.

In figure \ref{fig:qres_ex}, we can see a query where we asking for all the friends of the \texttt{BESTBOI} in our system. For each friend we ask for their \texttt{name}. The query can be further specified, using fragments, and say that for the \texttt{Dog} friends we want to know their toy's \texttt{chewiness} and for the \texttt{Pig} friends, their \texttt{oink} level. We rename this last selection to \texttt{loudness}. As we can see from this example, queries in \gql have a tree structure similar to \json.

If we evaluate this query in the graph depicted in \ref{fig:graph_ex}, we would get the response shown in figure \ref{fig:qres_ex}. This response was obtained by navigating the graph and collecting the information contained in each of the relevant nodes. It is easy to see that the response has a structure very similar to the query's.


\begin{figure}
\centering
\begin{subfigure}{.25\textwidth}
\begin{minted}[fontsize=\scriptsize, escapeinside=||,mathescape=true]{js}
query {
  goodboi(goodness:BESTBOI) {
    name
    friends {
      name
      |$\ldots$| on Dog {
        favoriteToy {
          chewiness
        }
      }
      |$\ldots$| on Pig {
        loudness:oink
      }
    }
  }
}
\end{minted}
\label{fig:query_ex}
\end{subfigure}%
\begin{subfigure}{.25\textwidth}
\begin{minted}[fontsize=\scriptsize]{json}
{
  "goodboi": {
    "name":"Casel",
    "friends":[
      {
        "name": "Marle",
        "favoriteToy": {
          "chewiness": 23
        },
      },
      {
        "name": "Chris P. Bacon",
        "loudness": 9000
      }
    ]
  }
}
\end{minted}
\label{fig:response_ex}
\end{subfigure}

\caption{Example of \gql query (left) and its response (right).}
\label{fig:qres_ex}
\end{figure}

If we wanted to ask another the same query but now without the friends' names, we would only have to remove the \texttt{name} field and \textit{voilà}, that's it. We use the same endpoint as before and the \gql service handles the resolution of our fields.

With this we conclude our brief introduction to \gql and we can now move onto the formalization. We will come back to this example throughout the rest of the paper, illustrating how it can be replicated in our system.

The key points to take from our example are ...
