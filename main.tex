\documentclass[sigplan,screen]{acmart}
\usepackage[utf8]{inputenc}
\usepackage{subcaption}
\usepackage{listings}
\usepackage{multicol}
\usepackage[cache=false]{minted}
\usepackage{listings}
\usepackage[nounderscore]{syntax}
\usepackage{graphicx}
\usepackage{xspace}
\usepackage{stmaryrd}
\usepackage{newunicodechar}
\newunicodechar{ə}{\pmschwa}
\DeclareRobustCommand{\pmschwa}{\rotatebox[origin=c]{180}{e}}
\usepackage{wrapfig}




%\linepenalty=20
%\clubpenalty=150
% \widowpenalty=150
%\parpenalty=-120


% Minted options
\setminted{fontsize=\footnotesize}
\definecolor{coqbg}{rgb}{0.95,0.95,0.95}


% Colors
\definecolor{darkgray}{HTML}{404040}
\definecolor{rulegray}{HTML}{DADADA}
\definecolor{keywordblue}{HTML}{1F497C}
\definecolor{mygray}{gray}{0.6}
% Listings
\lstdefinestyle{code}{
  basicstyle=\scriptsize,
  %backgroundcolor=\color{rulegray},
   numbers=left,
  emph=[1]{BEGIN, END, INPUT},
  emphstyle=[1]{\color{black}},
  keywordstyle=\color{keywordblue},
  keywords={LET, IN,CASE,IF,ELSE,THEN, WITH},
  escapeinside={(*@}{@*)}
}
% Listings
%\lstdefinestyle{code}{
  %basicstyle=\scriptsize,
  %keywordstyle=\color{keywordblue},
  %keywords={import,qualified,as,do,where}
%}


\newcommand{\mynote}[3]
    {{\color{#3} \fbox{\bfseries\sffamily\scriptsize#1}
    {\small$\blacktriangleright$\textsf{\emph{#2}}$\blacktriangleleft$}}~}
 \newcommand{\et}[1]{\mynote{ET}{#1}{purple}}
 \newcommand{\fo}[1]{\mynote{FO}{#1}{brown}}
 \newcommand{\td}[1]{\mynote{TD}{#1}{blue}}

% \newcommand{\et}[1]{}
% \newcommand{\fo}[1]{}
% \newcommand{\td}[1]{}


\newcommand\mdoubleplus{\mathbin{+\mkern-5mu+}}

% naming
\newcommand{\plstyle}[1]{\mbox{\textsc{#1}}\xspace}
\newcommand{\sparql}{\plstyle{SPARQL}}
\newcommand{\gql}{\plstyle{GraphQL}}
\newcommand{\gcoql}{\plstyle{GraphCoQL}}
\newcommand{\spec}{\plstyle{Spec}}
\newcommand{\HP}{\plstyle{H\&P}}
\newcommand{\json}{\plstyle{JSON}}
\newcommand{\jscert}{\plstyle{JSCert}}
\newcommand{\coqr}{\plstyle{CoqR}}
\newcommand{\coq}{\plstyle{Coq}}
\newcommand{\ssreflect}{\plstyle{Ssreflect}}
\newcommand{\mathcomp}{\plstyle{Mathematical Components}}
\newcommand{\equations}{\plstyle{Equations}}
\newcommand{\Vals}{$\mathit{Value}$\xspace}
\newcommand{\goodbois}{\plstyle{Artists}}
\newcommand{\movies}{\texttt{Movie}\xspace}
\newcommand{\fiction}{\texttt{Fiction}\xspace}
\newcommand{\animation}{\texttt{Animation}\xspace}
\newcommand{\artwork}{\texttt{Artwork}\xspace}
\newcommand{\artist}{\texttt{Artist}\xspace}

% math naming 
\newcommand{\query}{\varphi}
\newcommand{\sset}{\overline{\sigma}}
\newcommand{\sel}{\sigma}
\newcommand{\schema}{\mathcal{S}\xspace}
\newcommand{\graph}{\mathcal{G}\xspace}

\newcommand{\response}{\rho}
\newcommand{\ssize}[1]{\lvert #1 \rvert}

\newcommand{\name}{\mathit{name}}

% semantics
\newcommand{\filter}[2]{\mathit{filter(#2, #1)}}

\newcommand{\eval}[2]{\llbracket #1 \rrbracket^{#2}_{\graph}}
\newcommand{\evalu}[1]{\eval{#1}{u}}
\newcommand{\evalfilteru}[2]{\eval{\mathit{filter (#2, #1)}}{u}}
\newcommand{\collect}[3]{collect(#1, #2, #3)}
\newcommand{\ftype}[2]{ftype (#1, #2)}

% simpl semantics
\newcommand{\seval}[2]{\llparenthesis #1 \rrparenthesis^{#2}_{\graph}}
\newcommand{\sevalu}[1]{\seval{#1}{u}}

\newcommand{\queries}{\overline{\varphi}}
\newcommand{\subqueries}[1]{\overline{#1}}
\newcommand{\fkey}{\texttt{f}}

%selections
\newcommand{\args}{\overline{\alpha}}
\newcommand{\fld}{\texttt{f[}\args\texttt{]}}
\newcommand{\afld}{\texttt{a:}\fld}

 \newcommand{\nfld}[1]{\fld\; \texttt{\{}#1\texttt{\}} }
\newcommand{\anfld}[1]{\afld\; \texttt{\{}#1\texttt{\}} }

\newcommand{\ifrag}[2]{\texttt{... on #1 \{}#2\texttt{\}}}
\newcommand{\resp}[1]{\fkey\texttt{:}#1}
\newcommand{\nval}{\texttt{null}}
\newcommand{\val}{\texttt{v}}

%normalize
%\newcommand{\normalize}[1]{\mathit{normalize (#1)}}
\newcommand{\norm}[2]{\mathcal{N}_{\mathit{#1}}(#2)}
\newcommand{\tnorm}[1]{\norm{ts}{#1}}

%misc
\newcommand{\ie}{i.e.\@\xspace}
\newcommand{\wrt}{w.r.t.\@\xspace}
\newcommand{\eg}{e.g.\@\xspace}
\newcommand{\cf}{cf. \@\xspace}
\newcommand{\viz}{viz.\@\xspace}
\newcommand{\etal}{et al.\@\xspace}
\newcommand{\vscalar}{check\_scalar}


%\setcopyright{ACMUNKNOWN}
\copyrightyear{2020} 
\acmYear{2020} 
\setcopyright{acmlicensed}
\acmConference[CPP '20]{Proceedings of the 9th ACM SIGPLAN International Conference on Certified Programs and Proofs}{January 20--21, 2020}{New Orleans, LA, USA}
\acmBooktitle{Proceedings of the 9th ACM SIGPLAN International Conference on Certified Programs and Proofs (CPP '20), January 20--21, 2020, New Orleans, LA, USA}
\acmPrice{15.00}
\acmDOI{10.1145/3372885.3373822}
\acmISBN{978-1-4503-7097-4/20/01}



\begin{document}

\title{A Mechanized Formalization of \gql}

\renewcommand\footnotemark{}
\titlenote{This work is partially funded by ERC Starting Grant SECOMP (715753).}

\author{Tomás Díaz}
\affiliation{%
  \institution{IMFD Chile}
%  \city{Santiago}
%  \country{Chile}
}
\email{tdiaz@imfd.cl}

\author{Federico Olmedo}
\affiliation{%<
  \institution{Computer Science Department\\ University of Chile \&
    IMFD Chile}
%  \department{Computer Science Department (DCC)}
  % \institution{University of Chile \& IMFD Chile}
  % \department{Computer Science Department (DCC)}
%  \city{Santiago}
%  \country{Chile}
}
\email{folmedo@dcc.uchile.cl}

\author{Éric Tanter}
\affiliation{%
  \institution{Computer Science Department\\ 
  U. of Chile \&
    IMFD Chile \& Inria Paris}
  % \institution{University of Chile \& IMFD Chile}
  % \department{Computer Science Department (DCC)}
%  \city{Santiago}
%  \country{Chile}
}
\email{etanter@dcc.uchile.cl}

\begin{abstract}
\gql is a novel language for specifying and querying web APIs,
allowing clients to flexibly and efficiently retrieve data of
interest. The \gql language specification is unfortunately only
available in prose, making it hard to develop robust formal results
for this language. Recently, Hartig and Pérez proposed a formal
semantics for \gql in order to study the complexity of \gql
queries. The semantics is however not mechanized and leaves certain key aspects unverified. We present \gcoql, the first mechanized formalization of \gql, developed in the \coq proof assistant.  \gcoql covers the schema definition DSL, query definitions, validation of both schema and queries, as well as the semantics of queries over a graph data model.
We illustrate the application of \gcoql by formalizing the key query transformation and interpretation techniques of Hartig and Pérez, and proving them correct, after addressing some imprecisions and minor issues. 
We hope that \gcoql can serve as a solid formal baseline for both language design and verification efforts for \gql.
\end{abstract}

\begin{CCSXML}
<ccs2012>
<concept>
<concept_id>10002951.10003260.10003304</concept_id>
<concept_desc>Information systems~Web services</concept_desc>
<concept_significance>500</concept_significance>
</concept>
<concept>
<concept_id>10002951.10002952.10003197</concept_id>
<concept_desc>Information systems~Query languages</concept_desc>
<concept_significance>300</concept_significance>
</concept>
<concept>
<concept_id>10003752.10010124</concept_id>
<concept_desc>Theory of computation~Semantics and reasoning</concept_desc>
<concept_significance>500</concept_significance>
</concept>
</ccs2012>
\end{CCSXML}

\ccsdesc[500]{Information systems~Web services}
\ccsdesc[300]{Information systems~Query languages}
\ccsdesc[500]{Theory of computation~Semantics and reasoning}

\keywords{\gql, mechanized metatheory, \coq.}

\maketitle


%!TEX root = ./main.tex
\section{Introduction}

\gql is an increasingly popular language to define interfaces and queries to services data. Originally developed internally by Facebook as an alternative to RESTful Web Services, \gql was made public in 2015, along with a reference implementation\footnote{https://github.com/graphql/graphql-js} and a specification---both of which have naturally evolved since~\cite{gqlspec}\et{cite the version you used}. Since early 2019, as a result of its successful adoption by major players in industry,
\gql is driven by an independent foundation\footnote{https://foundation.graphql.org/}. The key novelty compared to traditional REST-based services is that tailored queries can be formulated directly by clients, allowing a very precise selection of which data ought to be sent back as response. This supports a much more flexible and efficient interaction model for clients of services, who do not need to gather results of multiple queries on their own, possibly wasting bandwidth with unnecessary data.
% many REST requests can be replaced by a single \gql query; additionally, and 
% that 

% follow a ``what you ask is what you get'' spirit. This means that, in contrast with REST-based services, one can be very precise with the data requested and the response will look very similar to the query.

The official \gql specification, called \spec hereafter, 
covers the definition of interfaces, the query language, and the validation processes, among other aspects. The specification undergoes regular revisions by an open working group, which meets monthly to discuss extensions and improvements, as well as addressing ambiguities. Indeed, the \spec is written in natural language, and does not include a rigorous formalization of \et{vague} its inner mechanics and limitations.
% related issues and improvements.  
% These include extending the language to support new features or fix possible ambiguities present in the document. This is because the document is written in natural language, i.e. plain English, 
% There is also a project to define CATs\footnote{Compatibility Acceptance Tests} for the different languages and frameworks that implement \gql\td{Not sure if this should go or where. It may serve as a link to "why of \gcoql"}.
Considering the actual vibrancy of the \gql community, sustained by several implementations in a variety of programming languages and underlying technologies, having a formal specification ought to bring some welcome clarity for all actors.

Recently, Hartig and Pérez~\cite{gqlph} proposed the first (and so far only) formalization of \gql, called \HP hereafter. 
\HP is a formalization ``on paper'' that was used to prove complexity boundaries for \gql queries. Having a mechanized formalization would present many additional benefits, such as potentially providing a faithful reference implementation, and serving as a solid basis to prove formal results about the \gql semantics. 

For instance, the complexity results of Hartig and Pérez rely on two techniques: {\em a)} transforming queries to {\em equivalent} queries in some  normal form, {\em b)} interpreting queries in a simplified but {\em equivalent} definition of the semantics. However, Hartig and Pérez do not prove that the query transformation (called {\em normalization}) indeed produces queries in such a normal form, and that their semantics is preserved; nor do they prove that the simplified semantics is equivalent to the original one, on such queries.

% The complexity results are based on two major premises. 
% The first one is that ``\textit{for every query $\varphi$ that conforms to a schema $\mathcal{S}$, there exists a {\normalfont non-redundant} query $\varphi$' in {\normalfont ground-typed normal form} such that $\varphi \equiv \varphi$'}''. The second one is that for queries that are \textit{non-redundant} and in \textit{ground-typed normal form}, it is possible to define a simplified version of the semantics which is equivalent to the original.

% For the former, they propose a set of equivalence rules to transform queries but they do not actually prove that their application yield a query in this particular form or that they preserve the query semantics. The latter is also exploited, without providing any correctness proof. Since both are fundamental for their complexity results, we believe they must be rigorously addressed.

\paragraph{\gcoql.} This work presents the first mechanized formalization of \gql, carried out in the \coq proof assistant \et{cite}, called \gcoql (|græf$\cdot$co$\cdot$k{\pmschwa}l|). In addition to precisely capturing the semantics of \gql, \gcoql makes it possible to completely specify and prove the correctness of query transformations, as well as other extensions and optimizations made to the language and its algorithms. We illustrate this by proving the correctness of \HP's normalization---in the process we uncover and address some imprecisions and minor issues.
We hope that \gcoql can serve as a starting point for a formal specification of \gql from which reference implementations can be extracted. Although we have not yet experimented with extraction, \gcoql facilitates this vision by relying on boolean reflection as much as possible.
% ; this should eventually make it possible to extract reference implementations of the different components developed in \gcoql, such as the query evaluator or the normalization function. 
 % and extracting it to be its official reference implementation.

%We will refer to it as \HP throughout the paper. They define the semantics of \gql by using a graph as the underlying data model over which queries are evaluated. \td{rewrite} define the normalization transformation, which results in \textit{non-redundant} queries in \textit{ground-typed normal form}.
% This normalization process is essential for proving complexity boundaries for \gql queries, because it allows simplifying the semantics and.

% On another note, we believe that \gql is still a very young and active technology which could greatly benefit by having its specification mechanically verified from its early stages. It has a very active and growing community, with many different implementations in different programming languages and technologies, and more importantly, with many open questions and issues. It currently has a reference implementation, written in Javascript, that could be improved by introducing a formally and mechanically verified one. %We refer to the reference implementation as \textit{\gqlJs} throughout the document.

 % Given the previous factors, we develop a Coq formalization of \gql, called 
 % \gcoql (pronounced ``græf$\cdot$co$\cdot$k{\pmschwa}l'').
 % We believe that \gcoql can serve as a starting point towards fully formalizing \gql and extracting it to be its official reference implementation.  

To address the trustworthiness of the \gcoql formalization, we have tried to establish a direct ``eyeball correspondence'' between \gcoql and the \spec whenever possible---though this correspondence has not (yet) been as seriously and systematically established as in the \jscert~\cite{jscert} and \coqr~\cite{coqr} projects, among others.
  % current status of \gcoql is less firmly 
  % following the examples of JSCert~\cite{jscert} and CoqR~\cite{coqr}.
 % This provides a component of trustworthiness given by an , 
 % We also test our implementation with examples from the \spec but a more thorough comparison should be made against \textit{\gqlJs} and a bigger test suite.
While \spec leaves the underlying data model unspecified \et{or under-specified?}, 
\HP adopts at its core a graph-based data model; \gcoql follows \HP in this regard, while the query evaluation algorithm of \gcoql can be traced closely to \spec.
\et{\gql is data model agnostic. \HP specializes to a graph-based data model. So do we. Readers will wonder: is this necessary? why? aren't we moving away from the actual \gql by taking this path? -- this needs some careful justification}

% \et{there is some contradiction between the eyeball correspondence and the "mixed approach"} 
% Like \HP, \gcoql  \et{what is the model in \spec?}
% With respect to the semantics of \gql, we follow a mixed approach between the \spec and \HP. The semantics are defined in a graph setting, as is in \HP, . 
% One of the biggest difference between both approaches (besides the graph model) is that the 
% \et{what is this processing about? and is the mixed approach you took here?}
% \spec performs a processing of queries during the evaluation, while \HP performs a post-processing of the responses generated. We took the mixed approach, which brings out some benefits as well as some limitations, which we discuss further in a following section.


%\td{Rewrite} When it comes to the underlying data model, we follow \HP and define our semantics in a graph setting. We are also interested in defining the properties and transformation rules defined by \HP. These definitions and their proofs of correctness are fundamental in the posterior results they obtain. This served as a particularly interesting first case study for our system, to establish that we can actually reason about \gql and that theirs results were based off correct assumptions. This allows us to, hopefully, anticipate that other transformations may also be defined and proven correct in \gcoql.



%We were first motivated to use it to define the data model and try to narrow our scope to finite types, as was used by (Veronique, Ev, Emilio, Dumbrava). In the end, we did not use any of it but the computational aspect of SSReflect was kept, as it facilitated developing the proofs. This same element is what makes us believe that extraction should not be hard.

\paragraph{Contributions.}
The contributions of this work are:
\begin{itemize}
    \item The first mechanized formalization of \gql (\S\ref{sec:form}), including the definition of the Schema DSL, query definition, schema and query validation, and the semantics of queries over a graph data model.
    \item An implementation of \HP's query normalization procedure, proven correct after fixing some imprecisions and minor issues (\S\ref{sec:norm}). We also formalize and prove the equivalence between original query evaluation semantics and the simplified semantics used by \HP when applied to normalized queries.
     % function with proofs of its correctness and preservation of semantics. This is a result used by \HP to prove complexity boundaries about \gql queries.
    % \item Proof of equivalence between the semantics and a simplified version. This is also an important result for posterior analysis made in \HP.
    \item \et{mention \S\ref{sec:discussion}}
\end{itemize}

We first briefly introduce \gql (\S\ref{sec:bg}). 
We end this article by discussing the validation and limitations of \gcoql (\S\ref{sec:valid}), related work (\S\ref{sec:related}) and conclude in \S\ref{sec:conclusion}.

\gcoql and the results presented in this paper have been developed in Coq v.\et{put version} and are provided as anonymous supplementary material. We often refer directly to Coq files in the paper. \gcoql extensively uses the 
\ssreflect \et{cite} and \equations \et{cite} libraries. 
The \gql specification on which this work is based is version\et{put version}.
% \et{I think we can skip this paragraph here, not important enough for being in the intro} Finally, regarding the development itself, we use SSReflect \et{cite} intensively, relying on boolean reflection as much as possible. Also, the use of the  to define non-structural recursive functions is essential for our definitions. Other libraries, such as \textit{Function} and \textit{Program} did not provide sufficient tools to handle rewriting and inductive reasoning about our definitions, which \textit{Equations} incredibly facilitates. \coql{} is not currently extracted to other languages but we believe that it should not be a difficult task, given the design decisions considered.

% \td{include note on code as anonymous supplementary material}


% \subsubsection*{Structure of this paper}

% We first begin by gently and briefly introducing \gql in Section \ref{sec:bg}, which we do by means of an example. Then, in Section \ref{sec:form}, we describe the basic building blocks of our Coq formalization. This includes the definition of a \gql schema, the graph data model, queries and their semantics. Section \ref{sec:norm} describes the normalization process and proofs of its correctness and preservation of semantics. We finalize that section with the definition of the simplified semantics, as described in \HP, and a proof of equivalence between the semantics defined in Section \ref{sec:form} and the simplified one. In Section \ref{sec:valid}, we describe some of the work we did to validate our implementation and finally Section \ref{sec:related} and \ref{sec:future} we discuss related and future work.


\section{Background}
\subsection{GraphQL Schema}
Describe what a graphQL Schema is, as per the spec. 

\subsection{GraphQL Query}
Describe what a GraphQL query is, as per the spec.

\subsection{GraphQL Response}

Describe what a GraphQL response is.

\subsection{Graph Data model}

Describe the graph model.



\section{GraphCoQL}\label{sec:form}

In this section we describe our formalization of GraphQL in Coq. We start by defining a schema and its properties, then the graph data model and finally we review queries and their semantics. The definitions are as close as possible with respect to the \spec{}. This eyeball correspondence between the english-written definitions and the code gives a first level of trust that our formalization is correct, following the examples of X, Y and Z. Whenever there is a mismatch we point it out and explain the reasoning behind each decision.

 \td{mention that we want to correlate to the spec and eyeball correspondence when possible.}

\td{The definitions consist of around 3700 loc and 1400 of lemmas}.

\subsection{GraphQL Schema}\label{subsec:schema}

The GraphQL schema is pretty straightforward to define from the grammar of the \spec{}. It consists of a collection of type definitions and a root query operation type. There is, however, a slight ambiguity when the \spec{} refers to the schema, as it is described as being ``\textit{defined in terms of the types and directives it supports as well as the root operation types for each kind of operation}''\footnote{https://graphql.github.io/graphql-spec/June2018/\#sec-Schema}. It then proceeds to define a structure called \texttt{schema} containing only the root operation types (query, mutation and subscription) and \textit{separately} it defines the type definitions, as well as the directives. The previously quoted definition actually matches the \textit{Type System} structure\footnote{https://graphql.github.io/graphql-spec/June2018/\#TypeSystemDefinition}. Our formalization follows the latter but rename it to schema to also match the quoted description.

\begin{minted}{coq}
Record graphQLSchema := GraphQLSchema {
    query_type : Name;
    type_definitions : seq TypeDefinition
}.
\end{minted}

Similarly, for type definitions we follow the grammar as specified in the \spec{}. Figure \ref{fig:types_def} shows the grammar and the corresponding implementation in Coq. As can be seen from the figure, our implementation looses information about non-emptiness of fields, union and enum members. We push this validation to a posterior predicate, as well as the discussion about the reasons behind this decision, to the following paragraphs.

% As can be seen in the figure, we tried to match the \spec{}'s definition as much as possible. This eyeball correspondence gives us a degree of confidence about the implementation.  % We currently do not include the \textit{Input Object} types, as well as anything related to \textit{introspection}.

Although the definitions are straightforward, both the \spec{}'s grammar and the Coq implementation allow building invalid schemas. For instance, it is possible to build an Object that implements scalar types or use a nonexistent type as the query type. To this end, the \spec{} includes validation rules scattered throughout the document\footnote{Most can be found in the \textbf{Type Validation} subsection of each type described in https://graphql.github.io/graphql-spec/draft/\#sec-Type-System.}. In \coql, we summarize these rules into predicates and refer to it as the \textit{well-formedness} property of a GraphQL schema. \HP{} refers to this property as the \textit{consistency} of the schema, to which we will refer briefly in a following paragraph.


\begin{definition}
A GraphQL schema is \textit{well-formed} if it satisfies the following conditions:
\begin{itemize}
    \item Its root query type is defined and is an Object type.
    \item There are no duplicated type names.
    \item Every type definition is \textit{well-formed}.
\end{itemize}
\end{definition}

The implementation in Coq is described by the following boolean predicate. As indicated in the introduction of this paper, we try to use boolean reflection as much as possible, following the SSReflect mindset.

\begin{minted}{coq}
Definition is_a_wf_schema (s : graphQLSchema) : bool :=
      is_object_type s s.(query_type) &&
      uniq s.(schema_names) &&
      all is_wf_type_def s.(type_definitions).
\end{minted}

Due to space constraints, we omit the definition of well-formedness for type definitions. The complete definitions can be found in the file \texttt{SchemaWellFormedness.v}. We will, though, resume the discussion about non-emptiness of fields, union and enum members, which are included in the predicate. The main reason behind this decision is that, even though the \spec{} embeds this information in the grammar, it still includes it in their validation rules later on. We believe that it is simpler to use common lists instead of defining new structures or using dependent types, from an implementation point of view, while still preserving the correspondence to the algorithmic description given by the \spec{}.
\td{Not sure if correctly worded... but it was just simpler to use lists. A non-empty list structure required coercions to lists and then redefining some lemmas and things. Or using dependent types (sigma type) adds complexity when proving and defining things (at least that was the case for me)}

Regarding \HP{}'s consistency property, they embed many properties in their structures, such as uniqueness of types given by using sets. They include an additional check on objects implementing interfaces, where they validate that fields are properly implemented. The definition given is not complete due to missing validation on arguments, but a corrected version is included in \cite{olafschema}.

% There are two main reasons why we push this rule to a separate predicate instead of embedding it in the structure itself. The first one is that, even though the \spec{} embeds it in the grammar, it still includes it in a validation rule later on. To match their definition and preserve the eyeball correspondence, we also include it. The second reason is that we use SSReflect and it is simpler to use \mintinline{coq}{seq} directly and all its theory, instead of defining coercions and repeating definitions for a new structure.


With the well-formedness property, we proceed to define a structure that encapsulates this notion, by passing both a schema and a proof of its validity.

\begin{minted}{coq}
Record wfGraphQLSchema := WFGraphQLSchema {
    schema : graphQLSchema;
    _ : schema.(is_a_wf_schema);
    is_a_valid_value : type -> Vals -> bool;
}.
\end{minted}

It is immediate that this structure requires an additional \mintinline{coq}{is_a_valid_value} predicate, which receives an element of \texttt{type} and a value of type \texttt{Vals}. This predicate is necessary to establish when a value used in a query or in the graph actually matches the scalar type expected by the schema. For instance, if an argument requires a \texttt{Float} value, then the actual value passed to the query must be something that represents a double-precision fractional value\footnote{The \spec{} declares a set of minimal scalar values and how they should be represented, such as floating-point values adhering to IEEE 754. We do not include this base restrictions but leave it open to implementation.}. This predicate validates that this is satisfied.

% Due to space constraints, we omit the definition of \textit{well-formedness} for type definitions. This property includes things such as: interfaces and objects must declare at least one field, objects correctly implement their declared interfaces, union types are not empty and contain only object types, amongst others. These definitions are collected from the \spec{} \td{Scattered throughout the \spec{}*}.

Finally, having defined the GraphQL schemas, we can move onto defining the data model used when evaluating queries.

\setlength{\grammarparsep}{20pt plus 1pt minus 1pt} % increase separation between rules
\begin{figure*}
    \centering
    \begin{subfigure}{.5\textwidth}
    \begin{grammar}
    <TypeDefinition> ::= \textbf{scalar} <name>
    \alt \textbf{type} <name> \textbf{implements} <name>* \textbf{\{} <Field>+ \textbf{\}}
    \alt \textbf{interface} <name> \textbf{\{} <Field>+ \textbf{\}}
    \alt \textbf{union} <name> \textbf{=} <name> \textbf{|} <name>*
    \alt \textbf{enum} <name> \textbf{\{} <name>+ \textbf{\}}

    <Field> ::= <name> \textbf{(} <Arg>* \textbf{) :} <type>

    <Arg> ::= <name> \textbf{:} <type>

    <type> ::= name
    \alt \textbf{[}  <type> \textbf{]}
    \end{grammar}

    \caption{Grammar of GraphQL types}
    \end{subfigure}%
    \begin{subfigure}{.5\textwidth}
    \begin{minted}{coq}
    Inductive TypeDefinition : Type :=
    | ScalarTypeDefinition (name : Name)

    | ObjectTypeDefinition (name : Name)
                           (interfaces : seq Name)
                           (fields : seq FieldDefinition)

    | InterfaceTypeDefinition (name : Name)
                              (fields : seq FieldDefinition)

    | UnionTypeDefinition (name : Name)
                          (members : seq Name)

    | EnumTypeDefinition (name : Name)
                         (members : seq EnumValue).

    Inductive type : Type :=
    | NamedType : Name -> type
    | ListType : type -> type.
    \end{minted}

    \caption{Implementation in Coq\td{Should include fields and arguments?}}
    \end{subfigure}
    \caption{Definition of GraphQL types.}
    \label{fig:types_def}
\end{figure*}



\iffalse
\begin{minted}{coq}
Let Animal := Interface "Animal" {[::
                (Schema.Field "name" [::] "String");
                (Schema.Field "friends" [::] ["Animal"])
            ]}.
Let Dog := Object "Dog" implements [:: "Animal"] {[::
            (Schema.Field "name" [::] "String");
            (Schema.Field "friends" [::] ["Animal"]);
            (Schema.Field "favouriteToy" [::] "Toy")
        ]}.
\end{minted}
\fi

\subsection{GraphQL Data model}\label{subsec:graph}

GraphQL is not tied to any particular database technology and implementation. When resolving fields in a query, GraphQL assumes the existence of \textit{resolvers}. These are internal functions defined by the user implementing a GraphQL service. They are not tied to any particular data model and the only requirement is that they must adhere to the schema. Whether they access a database, return static values or even modify existing data, is up to the user\footnote{The \spec{} states that these ``\textit{must always be side effect‐free and idempotent}'' but the definition of a resolver does not actually impose these restrictions.}. This makes reasoning about the semantics hard.

We choose to follow \HP{}'s approach and define the underlying data model as a graph over which queries are evaluated. With this model, the unspecified resolvers can be instantiated to concrete definitions which allow reasoning over them. The semantics are then described as being implemented over a graph setting. Although this provides benefits when reasoning about the semantics, it also comes with some potentially severe limitations over the completness of the possible results generated.\td{They may not actually be limitations with the model, but there are open questions on how to model some things.} We cover these limitations more thoroughly in a following paragraph and in Section~\ref{subsec:semantics}. It is worth mentioning that the limitations of this model are not described nor discussed in \HP{}.

Informally, a GraphQL graph is a directed property graph, with labeled edges and typed nodes. The graph describes entities with their types and properties, as well as the relationship between them. This means that every node has properties (key-value pairs) and a type. Also, every label in an edge describes the relation between two nodes. Finally, every property or label may also contain a list of arguments (key-value pairs).

We consider the type \Vals, representing the values associated to properties or used for arguments. A value in \Vals{} may be a single scalar value or a list of values.

\begin{definition}
A GraphQL graph over \Vals is defined by the following elements:
\begin{itemize}
    \item A root node.
    \item A collection of edges of the form ($u$, \texttt{f[}$\alpha$\texttt{]}, $v$), where $u, v$ are nodes and \texttt{f[}$\alpha$\texttt{]} is a label with arguments (key-value pairs).
\end{itemize}
\end{definition}

This is defined with the following structures in Coq.

\begin{minted}{coq}
Record fld := Field {
                  label : string;
                  args : seq (string * Vals)
                }.

Record node := Node {
                   ntype : Name;
                   nprops : seq (fld * Vals)
                 }.

Record graphQLGraph := GraphQLGraph {
                        root : node;
                        E : seq (node * fld * node)
                      }.
\end{minted}

\td{probably rewrite this paragraph...}
Our definition is in essence the same as in \HP{} but differs greatly in implementation. \HP{} defines a GraphQL graph in a more ``centralized'' manner. For instance, nodes and field names are defined by sets. Node types are defined by a single function which receives a node identifier and gives its type. Properties are also defined by a single function which receives a node identifier and a field name with arguments. Contrarily, our approach attempts to recreate the structures individually. For instance, a node contains all the information pertaining to itself; its type and its properties. We believe this is a more natural approach to defining the graph from an engineering point of view.

The definition of graph is completely independent of any GraphQL schema, so we need a way to relate the data to the type system. We implement the notion of \textit{conformance}  of a graph as partially described by \HP{}. This notion is, in essence, a well-formedness property for graphs with respect to a given schema. At the moment of development, there was no complete definition of conformance given by \HP{}. However, in a recent work by Hartig and Hidder's~\cite{olafschema}, they give a complete definition and extend it. Their approach uses a similar ``decentralized'' idea to define graphs. Their definitions capture more features than we currently implement, such as directives and non-null types.\td{And variables if I'm correct - should check}

\begin{definition}
A GraphQL graph \textit{conforms} to a schema $\mathcal{S}$ if it satisfies the following conditions:
\begin{itemize}
    \item The root node's type is equal to the query type.
    \item Every edge \textit{conforms} to $\mathcal{S}$.
    \item Every node \textit{conforms} to $\mathcal{S}$.
\end{itemize}
\end{definition}

This is captured in the following predicate in Coq.
\begin{minted}{coq}
Definition is_a_conforming_graph
        (s : wfGraphQLSchema)
        (graph : graphQLGraph) : bool :=

        root_type_conforms s g.(root) &&
        edges_conform s g &&
        nodes_conform s g.(nodes).
\end{minted}

Similarly to GraphQL schemas, we define a structure that encapsulates the notion of a \textit{conformed} graph. It contains a graph and a proof of its \textit{conformance} to a particular schema.

\begin{minted}{coq}
Record conformedGraph (s : wfGraphQLSchema) :=
            ConformedGraph {
                graph : graphQLGraph;
                _ : is_a_conforming_graph s graph
            }.
\end{minted}

Due to space limitations, we omit a detailed review of \textit{conformance} of nodes and edges. The complete definitions can be found in the file \texttt{GraphConformance.v}. %These properties include validation rules such as: every node must have an object type and their properties must be defined in their associated type, or an edge's label must be declared as a field in the source node's type and the target node must have a type compatible to the field's return type, among other things.

Finally, we partially retake the discussion on the limitations of this model. These have consequences on the semantics of GraphQL queries, so we delay some of it to the corresponding section. The main issue is that there is no proper accounting with respect to list types containing other list types (with any nesting depth). The different features that compose a GraphQL schema can be translated to a graph somehow. For instance, a field is either a property or the label of an edge, while its return type can be associated to a target node in an edge. However, when it comes to list types it is not clear what they represent in a graph. Let us illustrate this with an example.

A service may declare the field \texttt{friends:[Human]} in a given type, representing the list of friends.
In a graph this can be pictured as having a node with multiple outgoing edges labeled \texttt{friends}, reaching other nodes of type \texttt{Human}. It is possible to then extend the service by including a new field \texttt{friendsByName:[[Human]]}, in which one can request a list of friends but grouped by their names. At the moment neither our implementation, \HP{} nor \cite{olafschema} properly handle this situation. The open question is what does this represent in the graph? These should be outgoing edges similarly to the previous case but, what should the target nodes be? Should these be intermediate blank nodes? Is every edge labeled or only the last one that reaches a node with type \texttt{Human}? What happens if we increase the nesting? Since the information is ultimately collected from ``concrete'' nodes, should the graph be kept the same but introduce \textit{formatter} functions to match the schema?

These questions and more are not addressed nor discussed in \HP{} and it is actually more restrictive than expected, by not allowing nested lists for scalar values (in nodes's properties). Meanwhile, our approach and the one used in \cite{olafschema} allow any list type at the property level but simply ignore any possible nesting when the list type refers to neighboring nodes (composite types), as in the example above. In the case of \cite{olafschema}, they do not address nor discuss these questions. This choice of modeling has some consequences when defining the semantics of GraphQL queries, because the possible results generated are restricted to a smaller subset. It is not clear what the proper way is to handle this issue but more is explored in Section~\ref{subsec:semantics}. We also address the \spec{}'s semantics and how this is managed.

With both the schema and the underlying data model we can proceed to define GraphQL queries and their semantics.

\subsection{GraphQL Query}

As we mentioned in section \ref{sec:bg}, GraphQL queries are selections over types and fields defined in the schema. A GraphQL query can be seen as a tree structure where leaf nodes are selections of fields with a scalar return type. An inner node can be a selection on fields with an object or abstract return type. Inline fragments that condition when its subqueries are evaluated can also be seen as inner nodes. For instance, the query in Figure \ref{fig:qres_ex} can be depicted as the tree in Figure \ref{fig:query_tree}.

\begin{figure}
    \centering
    \includegraphics[scale=0.33]{imgs/query_tree.png}
    \caption{GraphQL query as a tree.}
    \label{fig:query_tree}
\end{figure}

Similar to the schema definition, we try to follow the \spec{}'s grammar as closely as possible. The grammar and implementation can be seen in Figure~\ref{fig:query_def}. There is a lost of information regarding non-emptiness of subqueries, as seen by rule 2 and the constructor \texttt{NestedField}. The reasoning behind this decision is very similar to the one used when implementing type definitions, which is described in Section~\ref{subsec:schema}.

\begin{figure*}
  \centering
  \begin{subfigure}{.5\textwidth}
    \begin{grammar}
        <Query> ::= <name> \textbf{(} <Arg>* \textbf{)}
        \alt <alias> \textbf{:} <name> \textbf{(} <Arg>* \textbf{)}
        \alt <name> \textbf{(} <Arg>* \textbf{)} \textbf{\{} <Query>+ \textbf{\}}
        \alt <alias> \textbf{:} <name> \textbf{(} <Arg>* \textbf{)} \textbf{\{} <Query>+ \textbf{\}}
        \alt \textbf{... on} <name> \textbf{\{} <Query>+ \textbf{\}}

        <Arg> ::= <name> \textbf{:} <value>
    \end{grammar}
    \caption{Grammar of GraphQL queries}
  \end{subfigure}%
  \begin{subfigure}{.5\textwidth}

    \begin{minted}{coq}
    Inductive Query : Type :=
    | SingleField (name : Name)
                  (arguments : seq (Name * Vals))

    | AliasedField (alias : Name)
                   (name : Name)
                   (arguments : seq (Name * Vals))

    | NestedField (name : Name)
                  (arguments : seq (Name * Vals))
                  (subqueries : seq Query)

    | NestedAliasedField (alias : Name)
                         (name : Name)
                         (arguments : seq (Name * Vals))
                         (subqueries : seq Query)

    | InlineFragment (type_condition : Name)
                     (subqueries : seq Query).
    \end{minted}
    \caption{Implementation in Coq}
  \end{subfigure}
  \caption{Definition of GraphQL queries.}
  \label{fig:query_def}
\end{figure*}


Both the \spec{} and our formalization differ from \HP{} when defining queries. The main difference is that \HP{} include an additional rule for lists of queries. Their grammar includes a production rule for lists of queries which is at the same level of the other rules. The main issue we found with this approach is that it allows building arbitrary trees instead of just a list of queries. These trees can be flattened to recover the list structure but this represents additional effort when defining functions and reasoning over queries. We believe this is assumed by \HP{} but not explicitly mentioned otherwise.


As in the case of well-formedness of schemas or conformance of graphs, queries must go through a validation process. We define the \textit{conformance} of queries based on validation rules scattered throughout the \textit{Validation} section of the \spec{}\footnote{https://graphql.github.io/graphql-spec/June2018/\#sec-Validation}.

Before defining the validation process, it is very important to address the notion of \textit{type in context} where selections are used. This notion is necessary to validate queries and when transforming queries, as described in Section~\ref{sec:norm}. The type in context is the type over which someone might be requesting information on its fields. For instance, in the following example the field selection \texttt{goodboi} is used in the context of the \texttt{Query} type. However, the type in the case of the field \texttt{name} is not entirely clear. In one case, the type in context is \texttt{Dog}, while in the other the field is used in the context of the \texttt{Pig} type.

\begin{minted}[escapeinside=||, mathescape=true]{gql}
      query {
        goodboi {
          |$\ldots$| on Dog {
            name
          }
          |$\ldots$| on Pig {
            name
          }
        }
      }
\end{minted}

The importance of this type in context is that fields or inline fragments might be valid in certain cases but not in others. Similarly, a field may have a particular return type in one case and a different one in another type, like in the following example. Both types have an \texttt{age} field, but in one case it returns an integer value while in the other a floating point value. If that field is encountered in a query, it is necessary to know to which type it is being requested.

\begin{minted}{graphql}
      type Human {
        age: Int
      }

      type Martian {
        age: Float
      }
\end{minted}


\begin{definition}
A GraphQL query $\varphi$ \textit{conforms} to a schema $\mathcal{S}$ if it satisfies the following conditions:
\begin{itemize}
    \item Selections in $\varphi$ are consistent.

    \item Field merging between fields is possible.
    % During the evaluation process, fields with the same response name are collected and merged to ensure that they are all executed at the same time. This validation rule checks that it makes sense to merge those fields. The following example illustrates two queries that have the same response name but should not be merged. The first one is accessing the field \texttt{name} while the second is accessing the field \texttt{age} but renaming it to \texttt{name}. Both are selections on different fields of the same type but with the same response name.
    % \begin{minted}{graphql}
    %                query {
    %                    name
    %                    name:age
    %                }
    %\end{minted}

    \item Fields with same response name have compatible response shapes.
    % This checks whether two fields with the same response name will produce response values that are consistent to each other. These values should be unambiguous for a user. For instance, the following example\td{These examples look a bit off I think.} shows two queries that produce similar responses but with ambiguous values. In the first one, we ask for dog's \texttt{name}s, which are strings, and in the second for pig's \texttt{age}s, which are integers. We also rename the \texttt{age} value to \texttt{name}. The responses we get will have some cases where \texttt{name} is associated to a string and other where it is associated to integers.
    % \begin{minted}[escapeinside=||,mathescape=true]{graphql}
    %                query {
    %                    |$\ldots$| on Dog {
    %                        name
    %                    }
    %                    |$\ldots$| on Pig {
    %                        name:age
    %                    }
    %                }
    % \end{minted}
\end{itemize}
\end{definition}

The definition in \coql{} is given by the following code. Due to space constraints, we do not include the complete definitions but they can be found in the file \texttt{QueryConformance.v}.

\begin{minted}{coq}
Definition queries_conform (type_in_scope : Name)
                           (queries : seq Query) : bool :=
        all (is_consistent type_in_scope) queries &&
        is_field_merging_possible type_in_scope queries &&
        have_compatible_response_shapes
            [seq (type_in_scope, q) | q <- queries].
\end{minted}

As described earlier, this rules are mostly a condensation of a set of validation rules defined in the \spec{}. The first one refers to whether a selection holds by itself. It includes checks such as: if query is over a field, then that field must be defined in the  type in context and its arguments are defined in the given field. Similarly, if a selection is an inline fragment, then the type condition has to be valid with respect to the type in context.

The second and third predicates are defined as a single validation rule in the \spec{}\footnote{https://graphql.github.io/graphql-spec/June2018/\#sec-Field-Selection-Merging}. We split them into two separate predicates because there is a chance for optimization. We noticed that the original definition includes redundant recursive calls which may result in increased computational time. At the time of writing this paper, a new algorithm was proposed by a team at XING\footnote{https://www.xing.com/} that also addresses this very same issue and is described in~\cite{xingalg}. They follow an approach using sets and provide a much more elaborate analysis of execution times than us. Comparing both approaches and analyzing execution times could be an interesting venue to explore.

During development, we also noticed that the \spec{}'s rule is too conservative and may consider valid queries as invalid. In a nutshell, the \spec{} allows defining fragments that are never evaluated. The issue is that the validation rule can then consider that subqueries in these fragments are invalid, even though they are never evaluated, rendering the whole query invalid\footnote{An example query can be seen in the following link: https://tinyurl.com/y3hz5vgv.}. The definition of the second predicate attempts to remove this conservativeness but we have not proved it. For the third predicate, we still have some conservative checks. Section \ref{subsec:invalidfrags} delves a little deeper into this issue.

% The main issue is that the \spec{} allows for what we call \textit{invalid fragments}, originally described in an issue in the \spec{}'s repository\footnote{https://github.com/graphql/graphql-spec/issues/367}. In a nutshell, the \spec{} allows using fragments with type conditions that can span to multiple unrelated types. These end up not being evaluated due to posterior checks\footnote{https://graphql.github.io/graphql-spec/June2018/\#DoesFragmentTypeApply()}.


Finally, with these definitions we can build queries in a GraphQL service. Examples may be found in the files \texttt{SpecExamples.v} and \texttt{HPExample.v}. From now on, we will assume that queries conform to a given schema. We can then move onto their semantics.

\subsection{Semantics}\label{subsec:semantics}

In this section we describe the semantics of GraphQL queries. We begin by briefly examining the responses generated by executing queries. Then we give an informal description of the semantics, followed by the formal definition. We finish by discussing some implementation choices and comparison with the \spec{} and \HP{}.

The \spec{} describes responses as a map. Our implementation differs slightly, modeling them as a tree structure, similar to JSON. We choose this structure to preserve similarity to queries and because it is simpler to preserve order of the responses. The \spec{} does not impose an ordering of responses, although encourages it\footnote{https://graphql.github.io/graphql-spec/June2018/\#sec-Serialized-Map-Ordering}. We believe that preserving the order is one of the selling points for GraphQL (queries and their responses are very similar and easy to read). Our approach has two main disadvantages: uniqueness of response names and cost of access. Since we use lists instead of maps, we can encounter duplicated names and accessing a value has a linear cost given by the lists size, instead of the constant access obtainable with a map. We still argue that the simplicity to obtain order is worth it. We do include a proof that the results obtained with the semantics have unique names. Finally, we use option types to represent null values in the leaves of the response tree.

\begin{minted}{coq}
Inductive ResponseNode (A : Type) : Type :=
| Leaf : A -> ResponseNode
| Object : seq (Name * ResponseNode) -> ResponseNode
| Array : seq ResponseNode -> ResponseNode.

Definition GraphQLResponse (Vals: eqType) :=
    seq (Name * (@ResponseNode (option Vals))).
\end{minted}

Moving onto the semantics of GraphQL queries. As we described in Section~\ref{subsec:graph}, the underlying data model is a graph, therefore the semantics are instantiated to this setting. In a following paragraph we briefly explore an alternative that is closer to the \spec{}, in the sense that it can be detached from a particular data model. In our setting a query then represents a navigation over a graph. At top level, a query starts from the root node and then moves around its edges and nodes, collecting data along the way. In this sense:
\begin{itemize}
    \item A field selection represents either accessing a node's property or traversing an edge to a neighboring node. On the neighboring nodes we recursively evaluate subqueries.
    \item An inline fragment conditions whether using a node to access its properties or to traverse to other nodes.
\end{itemize}

Figure \ref{fig:semantics} shows the formal definition of the semantics. It displays the cases where a field selection is accessing a node's property, when it is navigating to other nodes and when it is evaluating an inline fragment. Aliased fields are omitted for brevity but the complete definition can be explored in the file \texttt{QuerySemantics.v}.

The main difference with respect to \HP{} and the main similarity to the \spec{} is that we perform a collection of fields at the query level, whereas \HP{} performs a post-processing of responses. The main reasons are similarity to the \spec{} and difficulty in reasoning with \HP{}'s approach, which we explore in more depth in Section~\ref{subsec:semstories}.

%Our first attempt at defining the semantics was to follow \HP{}'s post-processing approach. Our intention was to be as close as possible to their formalization to later prove their transformation and equivalence results, which we cover in Section~\ref{sec:norm}. However, the non-structural recursive nature of both the transformations and the post-processing function made reasoning about semantic equivalence very hard.

\begin{figure*}
    \centering
    \begin{align}
    % Empty
    \eval{\cdot}{u} &= [\cdot] \\
    % SingleField
    \evalu{\fld\; ::\; \queries} &= \begin{cases}
    \resp{\val} \; ::\; \evalfilteru{\queries}{\fkey}  & \mathit{u.property}(\fld) = \val \\
    \resp{\nval} \; :: \; \evalfilteru{\queries}{\fkey} & \sim
    \end{cases}\\
    % Nested field
    \evalu{\nfld{\overline{\beta}} \; ::\; \queries} &=
    \begin{cases}
    \resp{\texttt{[} \mathit{map} (\lambda\; v_{i} \Rightarrow \eval{\overline{\beta} \mdoubleplus \mathit{merge (collect_\fkey (\queries))}}{v_{i}})\; \mathit{neighbors(u)} \texttt{]}} \; :: \; \evalfilteru{\queries}{f}  & \mathit{type(f)} \in L_{t} \text{and} \{v_{1}, \ldots, v_{k}\} = \{v_{i} \mid (u, f[\alpha], v_{i}) \in E\} \\
    (f:\{\eval{\subqueries{\beta}}{v}\})\; :: \; \evalfilteru{\queries}{f}  & \mathit{type(f)} \notin L_{t} \text{and} (u, f[\alpha], v) \in E \\
    (f:null)\; :: \; \evalfilteru{\queries}{f} & \mathit{type(f)} \notin L_{t} \text{and there is no } v \text{ s.t.} (u, f[\alpha], v) \in E \\
    \end{cases}\\
    %inline fragment
    \evalu{\ifrag{t}{\overline{\beta}}\; ::\; \queries} &= \begin{cases}
    \evalu{\overline{\beta} \mdoubleplus \queries} & \mathit{does\_fragment\_type\_apply_{\texttt{t}}(u.type)} = \texttt{true}\\
    \evalu{\queries} & \sim
    \end{cases}
    \end{align}
    \caption{Semantics for GraphQL queries.\td{This looks bad but I don't know how to format it :/}}
    \label{fig:semantics}
\end{figure*}

We finish this section by addressing two major aspects about our formalization; completeness and errors.

The first one was briefly mentioned in Section~\ref{subsec:graph}, when discussing the limitations and open questions regarding the graph model. These translate in the fact that we currently do not produce list results with nested lists of objects. For instance, the field \texttt{friendsByName:[[Human]]} is treated as if it were defined as \texttt{friendsByName:[Human]} and the results match the latter format. Otherwise, there is no restriction in the case of nested lists for scalar values. In \HP{}, there is no possibility to produce nested lists for either scalar or object values\footnote{The grammar itself does not permit it.} and there is no mention of this restriction.

Regarding error handling, we currently do not implement it. Errors may have two main sources; validation errors and execution errors.\td{Not sure how to write this}

%The first one is that we currently do not handle errors during execution. This is due to two main reasons: the evaluation function assumes it receives valid queries and we have not yet implemented non-null types. These relates to the two kinds of errors one may encounter when evaluating GraphQL queries: validation and execution errors. The first ones are captured before execution and displayed to the user. Our semantics has to deal with a case which would be ruled out by the validation process. We believe both cases can be covered by including X (monad/reasonably exceptional type theory/etc)\td{rewrite}.

% The second major aspect refers to completeness. Both our formalization and \HP{}'s do not generate all possible results expected by a GraphQL service. In particular, there is a limitation when generating lists with a nesting bigger than one. it does not generate results for list types of depth bigger than one, when its inner type is not a scalar type\footnote{HP goes a step further and does not allow any type of nested list result.}. For instance, one might want to get information about friends but grouped by their age. This could be modeled as a field with type \texttt{[[Human]]}, where the list type has depth 2. A response for this query would look something like \texttt{"friends":[[...], ..., [...]]}. This response cannot be generated by our semantics\footnote{It can be defined with the \mintinline{coq}{Response} structure but not generated with the semantics.}.

%The main challenge in this case is to define what this nested list types represent in a graph. If we take a simple case of a field with type \texttt{[Human]}, we can model it as neighbors of a node. However, if we increase the nesting such as \texttt{[[Human]]}, it becomes harder to model. What does this represent in the graph? Should we introduce blank nodes in between the source node and the \texttt{Human} nodes? Are these inner edges labeled? Should there be a blank node per each level of nesting or a single one with edges to itself? All these questions do not have a straightforward answer. Our semantics, as the one definded in PH, simply ignores any nesting bigger than one.\td{This is where it can be modelled using Functors. The \spec{} checks if it received a collection and applies map to eventually get to the concrete values. Not sure how to put this out there.}
\td{I am also missing functors and how the spec "should" be defined.}

This concludes the base formalization of GraphQL schemas, graph data model, and queries and their semantics. Using this basic structures we can start defining query transformations and prove some properties about them.

% !TEX root = ./main.tex
\section{Query Transformation: Normalization}\label{sec:norm}

To illustrate how \gcoql can be used to reason about query transformations, we study the {\em normalization} process proposed by Hartig and Pérez (\HP)~\cite{gqlph}, which is fundamental for the complexity results they prove.
%
Recall that these results are based on two premises: {\em a)} every query can be normalized to a semantically-equivalent query; {\em b)} on such queries, one can use a simplified, but equivalent, evaluation function. For normalization,  \HP provide a set of equivalence rules, which serve as rewriting rules\et{directedness?}\et{equivalence rules are not directed, rewriting rules should be (otherwise it wouldn't be terminating)}, but they do not prove the correctness of rewriting\et{explain: what does "correctness of rewriting" mean?}. Likewise, they do not prove the equivalence of the semantics when applied to normalized queries.

In this section, we use \gcoql to define the property of being in \textit{normal form}, the normalization procedure, as well as proving both correctness and semantic equivalence. Finally, we define a new simplified semantics and prove the equivalence to the original semantics from \S~\ref{subsec:semantics}.

It is worth mentioning that most of our formalization effort was devoted to defining and establishing the correctness of this normalization procedure. In terms of code, definitions are coded in approximately $350$ lines, while proofs amount to around $1,200$ lines. The complete definitions and proofs can be found in the files \texttt{QueryNormalForm.v} and \texttt{QueryNormalFormLemmas.v}. \et{does that include point b)?}\et{see the first paragraph of this section, I've introduced labels}

\subsection{Normal form}

The notion of \textit{normal form} introduced by Hartig and Pérez consists of the conjunction of two properties: being \textit{grounded} and being \textit{non-redundant}.
\HP refers to the former as being in \textit{ground-typed normal form}. %\et{remove/why is that confusing?} We believe this naming is confusing so we decide to rename it to the previously described property.

%\et{skip/move} Similarly to what is described in Section~\ref{subsec:query}, the normalization process requires information on the type in context where the queries might be defined. This is crucial as it guides the process on how queries are transformed.

% Throuhgout our development, we noticed that this definition given by \cite{gqlph} was too general when proving correctness of our normalization procedure. In particular, the definition states that the subqueries of a field selection can be either fields or fragments. This means that if there are two field selections with the same response name, one may have subqueries consisting of fields, while the other contains only inline fragments. This would cause issues when trying to remove redundancies in queries because one could not directly establish if the resulting queries satisfied the property.
\subsubsection*{Groundedness}
%\et{should it be "groundedness"? (being bounded is boundedness, being rounded is roundedness, etc.)}
Informally, the \textit{groundedness} property refers to whether queries are completely specified down to object types.  Consider the queries below, based on the schema from Figure~\ref{fig:schema_ex}\td{Might be a stretch to pull that figure from the beginning}, which request the name of an animal. The query to the left is not grounded because the field \texttt{name} is made over the abstract type \texttt{Animal} type. In contrast, the query on the right is requesting the same information, but it is grounded because it fully specifies the (concrete) object types on which it requests the information.
\td{Align to top or as it is? Aligning to top leaves a huge empty space :/}
\begin{minipage}{.25\textwidth}
\begin{minted}[escapeinside=||, mathescape=true]{js}
// Not grounded query
query {
  goodboi {
    name
  }
}
\end{minted}
\end{minipage}%
\begin{minipage}{.25\textwidth}
\begin{minted}[escapeinside=||, mathescape=true]{js}
// Grounded query
query {
  goodboi {
    |$\ldots$| on Dog {
	  name
    }
    |$\ldots$| on Pig {
      name
    }
  }	
}
\end{minted} 
\end{minipage}

%\et{verbose/repetitive} The main idea is that if a query is performed over an object type then the query should only be composed of field selections. In contrast, if the query is over an abstract type, then it should only be composed of inline fragments that specify the selections down to the object subtypes. In the former case, it does not make sense to use fragments to further specify a query because it is not possible to be more specific when querying an object. Meanwhile, in the latter case the query should clearly state what is being requested from each concrete subtype. 

\begin{definition}
A \gql query $\varphi$ is \textit{grounded} if it satisfies the following conditions, where \texttt{ty} is the type in scope.
\begin{itemize}
	\item If \texttt{ty} is an object type, then $\varphi$ contains only fields.
	\item If \texttt{ty} is an abstract type, then $\varphi$ contains only inline fragments whose type conditions are object types.
	\item Subqueries of $\varphi$ are \textit{grounded} w.r.t. to the field's return type or the fragments type condition.
\end{itemize}
\end{definition}

This definition of groundedness differs slightly from that of \HP \et{how is it different? why?}\td{See response in Zulip}\et{you can't just say that "it differs slightly" to the reader---you're  saying too much or too little. You can be approximate here and then clarify in the discussion section, or mention it here and saying that you'll expand on this in that later section}\et{I'd suggest to give you 2-3 lines to mention the main differences and put a forward ref to \S5)} nevertheless, we prove that our definition still implies being in ground-typed normal form \et{if you prove that it means you have also formalized the exact def from HP? so?}.

\begin{minted}{coq}
Lemma are_grounded_in_ground_typed_nf (s : wfGraphQLSchema)
                                      (type_in_scope : Name)
                                      (queries : seq Query) :
        are_grounded s type_in_scope queries ->
        are_in_ground_typed_nf s queries.
\end{minted}

\et{why this plural formulation?}\et{you must be kidding in your response...}\et{the question is why you don't have a lemma for a single query (and then if you need to check a sequence, you map that singular property over the sequence)}\et{notice that 4.1 talks about "*a* query is grounded"}
\et{type-in-scope is a (too) long variable name - you can use use "ts" throughout, just explaining clearly when you first use it}
%\et{watch out the argument was called type-in-scope while its used as ty.}

\et{why do we care about this lemma instead of the definition of is-grounded?}

\subsubsection*{Non-redundancy}

Informally, 
% \et{bad phrasing: what is "non-redundant" is a "query", not "whether there are no queries"} the notion of non-redundancy refers to whether there are no queries that may produce repeated results. 
a non-redundant query is a query that does not produce repeated results.
For example, consider the two queries below:
%\td{Should this be another figure?}\et{do the same as you do for groundedness -- better to just inline if there are no references from other parts of the paper}\et{use the same trick with comments to indicate grounded/ungrounded}

\begin{minipage}[t]{.25\textwidth}
\begin{minted}[escapeinside=||, mathescape=true]{js}
// Redundant query
query {
    goodboi {
        name
        name
        |$\ldots$| on Dog {
            name
        }
        |$\ldots$| on Dog {
            friends { |$\ldots$| }
}  }  }
\end{minted}
\end{minipage}%
\begin{minipage}[t]{.25\textwidth}
\begin{minted}[escapeinside=||, mathescape=true]{js}
// Non-redundant query
query {
    goodboi {
        |$\ldots$| on Dog {
            name
        }
        |$\ldots$| on Pig {
            name
        }
    } 
}
\end{minted} 
\end{minipage}
%\et{fix layout}

The query on the left requests the field \texttt{name} twice to the same type, and uses two fragments with the same type condition. If no collection and merger of fields is performed during the evaluation, this will produce two values with key \texttt{name}. Meanwhile, the query on the right requests information about each type only once. It is important to notice that, even though it requests the field \texttt{name} in both, this is considered not redundant because only one fragment will actually be executed at a time, 
depending on the concrete object value that is used to evaluate the query.

\begin{definition}
A \gql query $\varphi$ is \textit{non-redundant} if it satisfies the following conditions:
\begin{itemize}
    \item There is at most one field selection with a given response name, for a particular depth of the query tree.

    \item There is at most one inline fragment with a given type condition, for a particular depth of the query tree.
    
    \item Subqueries are non-redundant.
\end{itemize}
\end{definition}

Our definition of non-redundancy assumes that the queries are grounded \et{maybe the Coq definition, but not the definition above!}, much like \HP, mainly to simplify the implementation. The difficulty arises from using inline fragments and comparing their contents appropriately. This difficulty is further increased by the fact that the \spec currently allows using fragments that can possibly span over several unrelated types\footnote{https://graphql.github.io/graphql-spec/June2018/\#sec-Fragment-spread-is-possible}\footnote{https://github.com/graphql/graphql-spec/issues/367}\footnote{Example of inline fragments spanning to unrelated types - https://tinyurl.com/y4uxz3gu}\td{Should we expand on this somewhere?}.

\subsection{Normalization procedure}\label{subsec:normalization}

The normalization procedure can be understood as a form of abstract interpretation\et{before, tell us what it does}, which evaluates queries using only static information about the type in context. Figure~\ref{fig:normalize} displays pseudocode describing the process when the head of the list of queries is a field selection with subqueries. We do not include the complete definition because of space limitations, but the complete definition can be found in the file \texttt{QueryNormalForm.v}.


\begin{figure*}[h]
\centering
\begin{tabular}{c}
\begin{lstlisting}[
mathescape=true,
style=code]
(*@\textbf{BEGIN}@*) normalize
 (*@\textbf{INPUT}@*) schema, type_in_context, queries
 CASE queries WITH
  | nil => nil
  | CONS (response_name [args] { subqueries }) queries => 
    LET return_type := LOOKUP (schema, type_in_scope, response_name) IN
    LET collected := COLLECT (schema, response_name, type_in_scope, queries) IN
    LET merged := CONCAT (subqueries,  MERGE collected) IN
    LET filtered := FILTER (response_name, queries) IN
    
    IF IS_OBJECT_TYPE (schema, return_type) THEN 
     CONS
      response_name [args] { normalize (schema, return_type, merged)} 
      normalize (schema, type_in_scope, filtered)
    ELSE 
     LET subtypes := SUBTYPES (schema, type_in_scope) IN
     CONS
      response_name [args] { MAP ($\lambda$ subtype => INLINE (subtype, normalize (schema, subtype, merged))) subtypes } 
      normalize (schema, type_in_scope, filtered)
     
  | ...
(*@\textbf{END}@*) normalize
\end{lstlisting}
\end{tabular}
\caption{Pseudocode for the normalization procedure, showcasing field selections with subqueries.\td{Should we add brief defs. of COLLECT, MERGE, etc.?}}
\label{fig:normalize}
\end{figure*}

The complete normalization process is actually composed of two separate functions; \texttt{normalize}, which performs all the actual work but under the assumption that the type in context is an object type, and \texttt{normalize\_queries}, which makes no assumption but only pipes the work\et{?}\td{Don't know what else to add. One function does all the work, the other one just pipes the work the other, based on the type in context} \et{my question is because the expression "pipe the work" is not clear/correct} to the former. The main process can be informally described as consisting of two subprocesses that deal with the two aforementioned properties.\et{which? (merging, mentioned below, was not mentioned above)}
\iffalse
\begin{itemize}
    \item Grounding: Selections are either wrapped with inline fragments or lifted from an inline fragment.

    \item Merging: Fields with the same response name have their subqueries merged into a single selection.
    
    %Whenever a field is encountered, the procedure tries to find all fields with the same response name and merge their subqueries. It then proceeds to remove them from the list to ensure \textit{non-redundancy}. Comparing it to the the semantics, this is equivalent to the case when we evaluate a field and collect similar ones.
    
    %Since it is assumed that the type in context is an Object type, it will try to transform the query such that there are only fields left. This means it will try to get rid of inline fragments and lift their subqueries as much as possible. Much like if we were standing on a node in the graph, we only evaluate fragments and subqueries that make sense for that node's type (which is an Object type). In the case of fields, it will first check on its return type. If it is an abstract type, then it will create a cover of all possible concrete subtypes of the abstract type, by wrapping the subqueries with inline fragments. Otherwise, it will proceed recursively. Once again, this is like finding the neighbors of a node. Since a priori it doesn't know the neighboring nodes that may be encountered, the procedure anticipates all possible scenarios.
\end{itemize}
\fi

The first subprocess\et{name it} tackles\et{you "tackle" too much -- it's a vague term, be precise. What is "tackling the groundedness"? is it "transforming a query to an equivalent grounded one"? then say so. You can look for "tackle" in the whole paper and do the exercise of finding the thing tackle stands for.} the groundedness of queries, and corresponds to the \texttt{if-else} block \et{lines?} as well as the mapping in line 19 of Figure~\ref{fig:normalize}. The grounding is done by either wrapping selections with inline fragments, whenever the type in context is an abstract type, or by lifting nested selections from inside fragments, whenever their type conditions are compatible with the object type in context\td{This is related to the fact that \gql allows invalid fragments}. This process can be illustrated with the example for Figure~\ref{fig:grounded}. \et{there is no space for repetition: this is just saying that normalizing grounds, which we know already. So what is worth saying? (go the point, don't repeat)} Once again, the query to the left is not grounded since it requests the name of an animal without being specific down to the object types. The normalization process will then produce the query to the right, by wrapping the selection using fragments with the subtypes of the \texttt{Animal} type, namely \texttt{Dog} and \texttt{Pig}.

\iffalse
\td{This is the same example as above, so it can be reused}
\begin{minted}[escapeinside=||, mathescape=true]{js}
// Not grounded query
query {
    goodboi {
        name
    }
}
// Normalized query
query {
    goodboi {
        |$\ldots$| on Dog {
	    name
	}
	|$\ldots$| on Pig {
	    name
	}
    }	
}
\end{minted} 
\fi


\td{Unnecessary?} \et{yeah, I think all this is taking too much space for the actual technical content and insights it provides. Think of a more concise way to describe normalization, ie. by guiding the reader walk through FIgure 9}\et{examples are not necessary, you illustrated before when introducing the notions} \et{for now, I'm skipping the rest of 4.2, jumping to 4.3}
Next, to tackle inline fragments that are unnecessary or that specialize selections in the context of an object type, let us consider the following queries. The first query includes two fragments that are not necessary; a fragment with type condition \texttt{Query} and one with type condition \texttt{Animal}. The procedure eliminates both fragments, by lifting the subqueries.

\begin{minted}[escapeinside=||, mathescape=true]{js}
// Not grounded query
query {
    |$\ldots$| on Query {
        goodboi {
            |$\ldots$| on Dog {
	        |$\ldots$| on Animal {
		    name
		}
	    }
	}
}
// Normalized query
query {
    goodboi {
        |$\ldots$| on Dog {
	    name
	}
    }	
}
\end{minted} 

When it comes to removing redundancies in a query, there is a second subprocess that handles it, which roughly correspond to lines 7-9 in Figure~\ref{fig:normalize}. The process collects fields that share the same response name, merging their subqueries into a single selection. For example, the first query below is redundant since it requests the same \texttt{goodboi} field twice, and the subqueries in both cases also contain repeated \texttt{name} selections. The normalization process then merges the selections with the same response name, leaving only one occurrence of each case.

\begin{minipage}[t]{.25\textwidth}
\begin{minted}[escapeinside=||, mathescape=true]{js}
// Redundant query
query {
    goodboi {
        name
    }
    goodboi { 
 	name
    } 
}
\end{minted}
\end{minipage}%
\begin{minipage}[t]{.25\textwidth}
\begin{minted}[escapeinside=||, mathescape=true]{js}

// Normalized query
query {
    goodboi {
        |$\ldots$| on Dog {
	    name
	}
	|$\ldots$| on Pig {
	    name
	}
    }	
}
\end{minted} 
\end{minipage}

% With this definition, we define a second one, which makes no assumption on the type in context. This procedure only checks what kind of type it receives and either pipes the job to the previous one, or covers the queries with the possible concrete subtypes (and then pipes the work to the previous definition).

\iffalse
\begin{minted}{coq}
 Definition normalize_queries (s : wfGraphQLSchema)
                             (type_in_scope : Name)
                             (queries : seq Query) :
                                         seq Query :=
    if is_object_type s type_in_scope then
        normalize s type_in_scope queries
    else
        [seq on t { normalize s t queries } |
            t <- get_possible_types s type_in_scope].

\end{minted}
\fi

As a final note, it is worth mentioning that the subprocesses mentioned are not defined as separate functions, but occur interleaved in the normalization function. As a consequence, the definition is highly\td{?} non-structural but can be easily expressed using the Equations library. The  similarity between the normalization function and the semantics also eases reasoning about the preservation of the semantics. \td{Mention something about how we first split the definition into these 2 subprocesses but ended up being harder to reason about?}

We now move onto proving correctness of the normalization procedure and the preservation of the semantics.


% Finally, with these properties and definitions we prove the premises proposed by \HP. We leave that discussion, about \HP's approach and ours, to Section~\ref{subsec:discussion}.

\subsection{Proofs of correctness and preservation}

\et{the name of this "property/ies" keeps changing. Pick one, and use it throughout (ie. intro, beginning of 4)}

%As described initially in this paper \et{you recalled it in the header of 4, no need to say it again}, \HP base the complexity results over \gql queries on two premises; queries can be normalized, preserving their semantics, and there is an equivalent simplified function to evaluate queries in normal form. 
We now prove that any query can be normalized to an equivalent query, by 
proving the correctness of our normalization procedure. Recall that this key result is only assumed---not proven---by \HP.

\setcounter{equation}{0}% Restart equation counter
\begin{figure*}[h]
    \centering
    \begin{align}
    % Empty
    \seval{\cdot}{u} &= [\cdot] \\
    % SingleField
    \sevalu{\fld\; ::\; \queries} &= \begin{cases}
    \resp{\texttt{coerce(\val)}} \; ::\; \sevalu{\queries}  & \mathit{u.property}(\fld) = \val \\
    \resp{\nval} \; :: \; \sevalu{\queries} & \sim
    \end{cases}\\
    % Nested field
    \sevalu{\nfld{\overline{\beta}} \; ::\; \queries} &=
    \begin{cases}
    \resp{\texttt{[} \mathit{map} (\lambda\; v_{i} \Rightarrow\; \seval{\overline{\beta}}{v_{i}})\; \mathit{neighbors(u)} \texttt{]}} \; :: \; \sevalu{\queries}  & \mathit{type(f)} \in L_{t} \text{and} \{v_{1}, \ldots, v_{k}\} = \{v_{i} \mid (u, f[\alpha], v_{i}) \in E\} \\
    (f:\{\seval{\subqueries{\beta}}{v}\})\; :: \; \sevalu{\queries}  & \mathit{type(f)} \notin L_{t} \text{and} (u, f[\alpha], v) \in E \\
    (f:null)\; :: \; \sevalu{\queries} & \mathit{type(f)} \notin L_{t} \text{and there is no } v \text{ s.t.} (u, f[\alpha], v) \in E \\
    \end{cases}\\
    %inline fragment
    \sevalu{\ifrag{t}{\overline{\beta}}\; ::\; \queries} &= \begin{cases}
    \sevalu{\overline{\beta} \mdoubleplus \queries} & \mathit{does\_fragment\_type\_apply_{\texttt{t}}(u.type)} = \texttt{true}\\
    \sevalu{\queries} & \sim
    \end{cases}
    \end{align}
    \caption{Simplified semantics for queries in normal form.}
    \label{fig:simpl_semantics}
\end{figure*}


First, we prove that the normalization procedure is correct \et{meaning? there are tons of definitions of correctness, so you need to be super clear} and produces queries in normal form, by proving separately that the resulting queries are grounded and non-redundant\footnote{Since we prove that grounded implies ground-typed normal form, we can also prove ground-typed normal form for the normalization function.\et{this whole story about g-t normal form makes more noise than necessary, I think}. Both are proved by well-founded induction over the queries size\footnote{The notion of size includes the length of the list as well as the depth of the query tree.\et{this should be made clear in section 3, when you explain how you model queries. Just reading section 4, you jump between singular, list, and trees... confusing}} and auxiliary lemmas about subtyping.  We consider the case where the type in scope is an Object type \et{you use "object type" and "Object type" interchangeably, it's confusing} and the general case. 

\begin{minted}[escapeinside=||,mathescape=true]{coq}
Lemma normalize_are_grounded (s : wfGraphQLSchema)
                             (ty : Name)
                             (|$\varphi$| : seq Query) :
    is_object_type s ty ->
    are_grounded s ty (normalize s ty |$\varphi$|).

Lemma normalize_are_non_redundant (s : wfGraphQLSchema)
                                  (ty : Name)
                                  (|$\varphi$| : seq Query) :
    is_object_type s ty ->
    are_non_redundant (normalize s ty |$\varphi$|).

\end{minted}

Next, we prove normalization preserves the semantics of queries. To begin with, we prove the case where the type in context is the same as the type of the node where queries are being normalized. Lifting this to the top level, it corresponds to normalizing over the Query type and evaluating on the root node (whose type is the same, due to graph conformance). We then extend this notion to normalization over any type in context, \texttt{ty}\et{here you're using "ty" - ts or ty is good for me but be consistent throughout}, but with the restriction that the node's type must be subtype of \texttt{ty}. Once again, this is valid at top level over the Query type and the root node. Conformance of the graph also ensures that normalization and evaluation over neighboring nodes is preserved\et{why? I don't get it}. The proof also follows by well-founded induction over the queries size and auxiliary lemmas about graph conformance.

\begin{minted}[escapeinside=||,mathescape=true]{coq}
Lemma normalize_exec (s : wfGraphQLSchema)
                     (g : conformedGraph s)
                     (u : node)
                     (|$\varphi$| : seq Query) :
    u |\textbackslash|in g.(nodes) ->
    s, g |$\vdash$| |$\llbracket$| normalize s u.(ntype) |$\varphi$| |$\rrbracket$| in u with coerce =
    s, g |$\vdash$| |$\llbracket$| |$\varphi$| |$\rrbracket$| in u with coerce.

Theorem normalize_queries_exec (s : wfGraphQLSchema)
                               (g : conformedGraph s)
                               (u : node)
                               (ty : Name)
                               (|$\varphi$| : seq Query) :
    u |\textbackslash|in g.(nodes) ->
    u.(ntype) |\textbackslash|in get_possible_types s ty ->
    s, g |$\vdash$| |$\llbracket$| normalize_queries s ty |$\varphi$| |$\rrbracket$| in u with coerce =
    s, g |$\vdash$| |$\llbracket$| |$\varphi$| |$\rrbracket$| in u with coerce.

\end{minted}

This concludes our proofs of normalization and establish that the first premise used by \HP is correct in the context of our system.\et{you should mention the problems you discovered about \HP by doing this formalization}
% The next section continues with the second premise, namely the definition of a simplified version of the semantics and the proof of equivalence.

\subsection{Simplified semantics}

As proposed by \HP, one of the main properties of queries in normal form is that they produce non-redundant responses\et{is that the main motivation for HP to propose normalization in the first place? (if so, say it)}\td{They actually say "unique responses" - they don't define non-redundancy for responses.}\et{why change the name? unique or non-redundant?}, without the need of any collection and merger of fields. This property allows defining a second evaluation function $\ll \varphi \gg_{G}$, similar to the original (\S\ref{subsec:semantics}), but without any filtering and collecting of fields. Figure~\ref{fig:simpl_semantics} shows the simplified semantics's formal definition. Aliased cases\et{?} are not included due to space constraints.
\HP do not formally prove however that the simplified semantics are equivalent to the original, when considering normalized queries.

We define the simplified semantics of \HP, and prove that, for queries in normal form, both $\llbracket \varphi \rrbracket_{G}$ and $\ll \varphi \gg_{G}$ produce the same response. We do not include the complete definition---they can be found in files \texttt{QuerySemantics.v} and \texttt{QuerySemanticsLemmas.v}. The proof is once again performed by induction over the size of the queries. \et{any insight from the proofs?}

\begin{minted}[escapeinside=||,mathescape=true]{coq}
Theorem exec_equivalence (s : wfGraphQLSchema)
                         (g : conformedGraph s)
                         (u : node)
                         (|$\varphi$| : seq Query) :
    are_in_ground_typed_nf s |$\varphi$| ->
    are_non_redundant |$\varphi$| ->
    s, g |$\vdash$| |$\llbracket$| |$\varphi$| |$\rrbracket$| in u with coerce =
    s, g |$\vdash$| |$\ll$| |$\varphi$| |$\gg$| in u with coerce.
\end{minted}

\et{not useful conclusion paragraph---and not finished} This result concludes the normalization process. We have described our approach to proving both premises exploited by \HP. As mentioned earlier, these are fundamental for their complexity results over \gql queries but were not proven true. We believe our approach properly and rigorously addresses them. The following section discusses

% !TEX root = ./main.tex
\section{Discussion}\label{sec:discussion}

\subsection{Schema}

There is, however, a slight ambiguity when the \spec{} refers to the schema, as it is described as being ``\textit{defined in terms of the types and directives it supports as well as the root operation types for each kind of operation}''\footnote{https://graphql.github.io/graphql-spec/June2018/\#sec-Schema}. It then proceeds to define a structure called \texttt{schema} containing only the root operation types (query, mutation and subscription) and \textit{separately} it defines the type definitions, as well as the directives. The previously quoted definition actually matches the \textit{Type System} structure\footnote{https://graphql.github.io/graphql-spec/June2018/\#TypeSystemDefinition}. Our formalization follows the latter but rename it to schema to also match the quoted description.

Regarding \HP{}'s consistency property, they embed many properties in their structures, such as uniqueness of types given by using sets. They include an additional check on objects implementing interfaces, where they validate that fields are properly implemented. The definition given is not complete due to missing validation on arguments, but a corrected version is included in \cite{olafschema}.

\subsection{Data model}


Our definition is in essence the same as in \HP{} but differs greatly in implementation. \HP{} defines a GraphQL graph in a more ``centralized'' manner. For instance, nodes and field names are defined by sets. Node types are defined by a single function which receives a node identifier and gives its type. Properties are also defined by a single function which receives a node identifier and a field name with arguments. Contrarily, our approach attempts to recreate the structures individually. For instance, a node contains all the information pertaining to itself; its type and its properties. We believe this is a more natural approach to defining the graph from an engineering point of view.


Finally, we partially retake the discussion on the limitations of this model. These have consequences on the semantics of GraphQL queries, so we delay some of it to the corresponding section. The main issue is that there is no proper accounting with respect to list types containing other list types (with any nesting depth). The different features that compose a GraphQL schema can be translated to a graph somehow. For instance, a field is either a property or the label of an edge, while its return type can be associated to a target node in an edge. However, when it comes to list types it is not clear what they represent in a graph. Let us illustrate this with an example.

A service may declare the field \texttt{friends:[Human]} in a given type, representing the list of friends.
In a graph this can be pictured as having a node with multiple outgoing edges labeled \texttt{friends}, reaching other nodes of type \texttt{Human}. It is possible to then extend the service by including a new field \texttt{friendsByName:[[Human]]}, in which one can request a list of friends but grouped by their names. At the moment neither our implementation, \HP{} nor \cite{olafschema} properly handle this situation. The open question is what does this represent in the graph? These should be outgoing edges similarly to the previous case but, what should the target nodes be? Should these be intermediate blank nodes? Is every edge labeled or only the last one that reaches a node with type \texttt{Human}? What happens if we increase the nesting? Since the information is ultimately collected from ``concrete'' nodes, should the graph be kept the same but introduce \textit{formatter} functions to match the schema?

These questions and more are not addressed nor discussed in \HP{} and it is actually more restrictive than expected, by not allowing nested lists for scalar values (in nodes's properties). Meanwhile, our approach and the one used in \cite{olafschema} allow any list type at the property level but simply ignore any possible nesting when the list type refers to neighboring nodes (composite types), as in the example above. In the case of \cite{olafschema}, they do not address nor discuss these questions. This choice of modeling has some consequences when defining the semantics of GraphQL queries, because the possible results generated are restricted to a smaller subset. It is not clear what the proper way is to handle this issue but more is explored in Section~\ref{subsec:semantics}. We also address the \spec{}'s semantics and how this is managed.

\subsection{Queries}


Both the \spec{} and our formalization differ from \HP{} when defining queries. The main difference is that \HP{} include an additional rule for lists of queries. Their grammar includes a production rule for lists of queries which is at the same level of the other rules. The main issue we found with this approach is that it allows building arbitrary trees instead of just a list of queries. These trees can be flattened to recover the list structure but this represents additional effort when defining functions and reasoning over queries. We believe this is assumed by \HP{} but not explicitly mentioned otherwise.


The second and third predicates are defined as a single validation rule in the \spec{}\footnote{https://graphql.github.io/graphql-spec/June2018/\#sec-Field-Selection-Merging}. We split them into two separate predicates because there is a chance for optimization. We noticed that the original definition includes redundant recursive calls which may result in increased computational time. At the time of writing this paper, a new algorithm was proposed by a team at XING\footnote{https://www.xing.com/} that also addresses this very same issue and is described in~\cite{xingalg}. They follow an approach using sets and provide a much more elaborate analysis of execution times than us. Comparing both approaches and analyzing execution times could be an interesting venue to explore.


During development, we also noticed that the \spec{}'s rule is too conservative and may consider valid queries as invalid. In a nutshell, the \spec{} allows defining fragments that are never evaluated. The issue is that the validation rule can then consider that subqueries in these fragments are invalid, even though they are never evaluated, rendering the whole query invalid\footnote{An example query can be seen in the following link: https://tinyurl.com/y3hz5vgv.}. The definition of the second predicate attempts to remove this conservativeness but we have not proved it. For the third predicate, we still have some conservative checks. Section \ref{subsec:invalidfrags} delves a little deeper into this issue.


\subsection{Semantics}

We finish this section by addressing two major aspects about our formalization; completeness and errors.

The first one was briefly mentioned in Section~\ref{subsec:graph}, when discussing the limitations and open questions regarding the graph model. These translate in the fact that we currently do not produce list results with nested lists of objects. For instance, the field \texttt{friendsByName:[[Human]]} is treated as if it were defined as \texttt{friendsByName:[Human]} and the results match the latter format. Otherwise, there is no restriction in the case of nested lists for scalar values. In \HP{}, there is no possibility to produce nested lists for either scalar or object values\footnote{The grammar itself does not permit it.} and there is no mention of this restriction.

Regarding error handling, we currently do not implement it. Errors may have two main sources; validation errors and execution errors.\td{Not sure how to write this}

% % !TEX root = ./main.tex


% We briefly mention the features that are not currently implemented in \gcoql. 
% Each feature will point to the corresponding section or sections 

% To begin with, the missing executable definitions are:
% \begin{itemize}
% 	\item Mutations (\cf\S2.3).
% 	\item Subscriptions (\cf\S2.3).
% 	\item Fragments (\cf\S2.8).\td{This is different from inline fragments !}
% \end{itemize}

% Regarding the possible operations and definitions on types, the following are not defined.
% \begin{itemize}
% 	\item Non-null types (\cf\S2.11, 3.4.1, 3.12). 
% 	\item Type System extensions, either schema or type extensions (\cf\S3.2.2, 3.4.3).
% 	\item Arguments default values (\cf\S3.6.1).
% 	\item Input object types (\cf\S3.10).
% 	\item Directives (\cf\S3.13) 
% \end{itemize}
% \td{There are directive definitions (which fall into type definition) and directive usage (which are in selections...)}

% Then, the features related to queries and their execution that are not currently supported.
% \begin{itemize}
% 	\item Input object values (\cf\S2.9.8).
% 	\item Variables (\cf\S2.10).
% 	\item Variables and arguments coercion (\cf\S6.1.2, 6.4.1).
% 	\item Normal and serial execution (\cf\S6.3.1)
% 	\item Error handling (\cf\S6.4.4). 
% \end{itemize}

% Finally, there is nothing on the \textit{Introspection} (\cf\S4) system. 



% %!TEX root = ./main.tex
\section{Validation}\label{sec:valid}
%\td{This was supposed to be a mention to the examples implemented in our system - HP, spec's, etc}

%\et{expand a bit on what you have to say about the examples}

To validate that \gcoql adequately captures the semantics of \gql, in additional to establishing an eyeball correspondence with \spec, 
we implemented several examples, coming from three different sources.
\et{you say four sources, but below there is only three}
\td{Ah, lol. The fourth one was the one we use in here -- although it is basically the same as the one in \HP}

First, we implement all of the examples used in the \spec validation section (\cf\S5~\cite{gqlspec}), for features that \gcoql currently supports.
These correspond to tests over fields, such as valid definition in given types in context, proper use of arguments and 
whether they are are type-compatible and renaming-consistent. Also, the examples include validation over inline fragments and 
whether they can be used in certain contexts. 


%We include 41 of the 92 total examples present in that section.\et{this is not a very good number. Can you get it up by the deadline? or give better evidence that the subset includes the most important ones?}
%The large majority of the examples that we have not implemented correspond to 
%variables validation (31 cases) and fragments (29)\td{Some of these are tests that overlap with the inline fragments -- So we are still kind of testing them}
%\td{The cases that are exclusive to fragments are 8}\td{Also, some cases use fragment only as a way to write the tests, but they are not really necessary
%(they are not testing things from the fragments) so I don't consider them}.
%\td{The rest of cases are a mix of things -- 11 are related to operations (\eg no two queries with the same name, no named query + another unnamed query, etc)}

\et{relate to the limitations below: of course you can't include examples that require unsupported features. Is it as simple as saying that you have implemented all examples that do not require such features?}
\td{Yes}
\et{move limitations section to above, and allude to limitations when discussing what has (not) been included in the tests.}
\et{try to put some "positive characterization" of what the implemented examples cover}

Second, we implement the example used in \HP, from its schema to its graph, query and corresponding response. 
We define the values in $\mathit{Vals}$ as elements of an inductive type, which wraps already defined types in Coq
(such as integer, string, etc.), and a coercion function to transform them into the matching JSON format.\td{This is actually done in all examples.}




\iffalse
	-	Only executable definitions when evaluating = -1 (text + query)
	- Uniqueness of operation = -3
	- Anonymous operation = -2
	- Suscription = -5 (single root field)
	- Fields = 2 + 2 + 2 - 1 (introspection on union)
	- Field merging = 3 + 1 - 1 (variable) + 2 - 2  (variables) + 2  + 1
	- Leaf fields = 2 + 3 - 1 (type ext)
	- Argument names = 2 - 1 (directives) + 1 - 1 (directives) + 2  - 1 (type extension)
	- Required args = -5 (required arguments - non-null + default values)
	- Fragment uniqueness = -2 (fragments)
	- Fragment spread type existence = 1 - 3 (2 fragments y 1 directive) + 1 - 2 (fragments)
	- Fragments on comp types = 3 - 3 (fragments) + 2 - 2 (fragments)
	- Fragments must be used = -1
	- Fragments spread = -4 (fragments) + 2 - 2 (fragments) + 1 - 2 (fragments) + 1 - 2 (fragments) + 2 - 2  + 2 - 2 + 1 - 2 + 1 - 2 
	- Valid Input values = -5
	- Input objects = -3
	- Directives = -3
	- Variables = - 2 - 8 - 7 - 4 - 7
	
	Hechos : 2 + 2 + 2 + 3 + 1 + 2 + 2 + 2 + 3 + 2 + 1 + 2  + 1 + 1 + 3 + 2 + 2 + 1 + 1 + 2 + 2 + 1 + 1 = 41
	No hechos : -1 - 3 - 2 - 5 - 1 - 1 - 2 - 1 - 1 - 1 - 1 - 5 - 2 - 3 - 2 - 3 - 2  - 1 - 2 - 2 - 2 - 2 - 2 - 2 -2 - 2 - 5 - 3 - 3 - 2 - 8 - 7 - 4 - 7  = 92
	- Variables = -1 - 2 - 2 - 8 - 7 - 4 - 7 = 31
	- Fragments = -2 - 2 - 2 - 3 - 2 - 1 - 4 - 2 - 2 - 2 - 2 - 2 - 1 - 2 = 29 
	- Operations = -1 - 3 - 2 - 5 = 11
	- Directives = -1 -1 - 1 - 3 = 6
	- Valid input values = 5
\fi 
	
Finally, we also partially\et{is it only the error stuff that's missing?}\td{Same as above, it is missing not implemented features (errors, fragments, etc.)} implemented the Star Wars example defined in the reference implementation of \gql\footnote{https://github.com/graphql/graphql-js/tree/master/src/\_\_tests\_\_}.
We include the schema definition, except for the \texttt{secretBackstory} field which resolves to an error\td{Bc we are not handling errors}. The data modeled as a graph and 
evaluation of 9 queries (out of a total of 17), comparing them to the expected outputs. Of the not implemented cases, 3 correspond to errors, 2 to introspection, 1 to fragments, 2 to variables and 
one where execution is over an object with named properties\td{Idk, something more specific to JS than \gql}.
\td{There are 7 other about validation, which I didn't notice, but I can add them}
\td{In that same repo (https://github.com/graphql/graphql-js/tree/master/src) each subfolder (type, execution, etc) contain other \_\_tests\_\_ folders, but
I didn't delve too much into them -- some are a bit too particular to JS. They can still be adapted, but not now}
		

% Most of the development time was spent in the definition and proofs of normalization.
% We initially worked on the semantics as specified by \cite{gqlph}

%\et{have a brief section on limitations that lists everything from the spec that is not covered currently in \gcoql (operations beyond query, errors, etc)}


%!TEX root = ./main.tex
\section{Related Work}\label{sec:related}

% \paragraph{\gql.} 
To the best of our knowledge, the only formalization efforts around \gql are \HP~\cite{gqlph} and~\cite{olafschema}, which we have already discussed. The rest of the \gql literature focuses on practical issues such as creating \gql services and migrating REST-based web services \gql~\cite{improvingoeeu, ehriapi, gqlexperiences}, automatic migration~\cite{migratingapi}, and testing techniques~\cite{gqldeviation}. A couple of empirical studies analyze the structure of \gql schemas in commercial and open-source projects~\cite{empiricalgql, empiricalapi}, shedding interesting insights on \gql in practice.
% Although these provide , they do not provide guarantees \wrt the specification's correctness. 
% As far as we are aware, the only works on formalizing \gql are the present work and~\cite{gqlph}.
% Most of the academic research on \gql 
% Others focus on testing techniques for implemented \gql services or tools for 

% \paragraph{Mechanization of query languages.} 
The formalization of data management systems has received a lot of attention in the traditional 
relational data model~\cite{relationalcoq}, SQL and its semantics~\cite{sqlequiv, hottsql, vesqlengines, vesqlsemantics}, as well as related query languages and engines~\cite{certifdatalog}. The tree-based nature of \gql queries and response differs significantly from the tuple-based semantics in traditional query languages, requiring different models and techniques.
% Although these efforts extensively cover these models and languages, we believe they are 
% not directly applicable to the \gql setting\td{One of the reasons is that queries and responses in \gql are both trees (the atomic response IS a tree), while }. 
% \td{Idk what to say tbh. I believe most of this work doesn't directly apply to our context, but it exists and cover several topics.}
% \td{Also, I am not an expert in databases and query languages, so I am not entirely sure how or what to compare}
% \td{I remember having a talk with Jorge where he mentioned that \gql is a lot simpler and that, although he thought they can be related to AQL, 
% they might not, because of this tree form of queries and responses -- which differs completely from the list of tuples in other query languages}
% \paragraph{Mechanization of graph data model.} 
Doczkal and Pous~\cite{graphtheory} develop a mechanization of graph theory in \coq, including simple graphs, digraphs, and their properties. Their work could possibly be extended to deal with property graphs, and used for \gcoql.
Bonifati \etal~\cite{graphviewmaint} study graph view maintenance. Their definition of graphs is tailored to handle regular Datalog queries and does not entirely fit the \gql setting; it could however serve as a base to extend \gcoql with mutation.

% There is also work on 
% \td{They define a function that given a regexp returns a whole graph of the nodes that are connected via the regexp. Nodes are not typed, nor do they have properties I believe (not sure).}

% \paragraph{\gql miscellaneous.} 
Finally, despite being used mostly for web services, there are efforts to extend notions used in \gql to other areas of database specification and querying. Hartig and Hidders~\cite{olafschema} use the \gql schema definition DSL to define the structure of property graphs, which can be linked to similar efforts to define the structure of graph databases~\cite{schemaval}. Taelman \etal study the transformation of \gql queries to SPARQL~\cite{gqlsparql}.\et{is there any mechanized formalization of sparql?}\td{Haven't found one yet!}
% using \gql for deductive database in a Prolog setting~\cite{gqldeductive}. 
% \td{Not entirely sure what the last one is tbh, so probably	kill xd}


\td{Mention the JSON formalization done by reutter?~\cite{json}}\et{what could we do with it? is there anything interesting to say? (eg. for responses?)}
\td{Not sure. I looked at it when I was confused with HP's responses and thought I could base it off of it, but their formalization is a bit different than what I was expecting.
They mention that theirs is the first formalization, but idk}


\section{Conclusions}
\label{sec:conclusion}

\et{what has been achieved}


\et{future work}

Extraction.

Testing and comparing with ref implementation.

Automation of proofs

Extend to include more things (handle errors, handle variables, etc.)

Collab with GraphQL foundation/community


% \clearpage
\bibliographystyle{acm}
\bibliography{biblio}
\end{document}
