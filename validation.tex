%!TEX root = ./main.tex
\section{Validation}\label{sec:valid}
%\td{This was supposed to be a mention to the examples implemented in our system - HP, spec's, etc}

%\et{expand a bit on what you have to say about the examples}

To validate that \gcoql adequately captures the semantics of \gql, in additional to establishing an eyeball correspondence with \spec, 
we implemented several examples, coming from three different sources.
\et{you say four sources, but below there is only three}
\td{Ah, lol. The fourth one was the one we use in here -- although it is basically the same as the one in \HP}

First, we implement all of the examples used in the \spec validation section (\cf\S5~\cite{gqlspec}), for features that \gcoql currently supports.
These correspond to tests over fields, such as valid definition in given types in context, proper use of arguments and 
whether they are are type-compatible and renaming-consistent. Also, the examples include validation over inline fragments and 
whether they can be used in certain contexts. 


%We include 41 of the 92 total examples present in that section.\et{this is not a very good number. Can you get it up by the deadline? or give better evidence that the subset includes the most important ones?}
%The large majority of the examples that we have not implemented correspond to 
%variables validation (31 cases) and fragments (29)\td{Some of these are tests that overlap with the inline fragments -- So we are still kind of testing them}
%\td{The cases that are exclusive to fragments are 8}\td{Also, some cases use fragment only as a way to write the tests, but they are not really necessary
%(they are not testing things from the fragments) so I don't consider them}.
%\td{The rest of cases are a mix of things -- 11 are related to operations (\eg no two queries with the same name, no named query + another unnamed query, etc)}

\et{relate to the limitations below: of course you can't include examples that require unsupported features. Is it as simple as saying that you have implemented all examples that do not require such features?}
\td{Yes}
\et{move limitations section to above, and allude to limitations when discussing what has (not) been included in the tests.}
\et{try to put some "positive characterization" of what the implemented examples cover}

Second, we implement the example used in \HP, from its schema to its graph, query and corresponding response. 
We define the values in $\mathit{Vals}$ as elements of an inductive type, which wraps already defined types in Coq
(such as integer, string, etc.), and a coercion function to transform them into the matching JSON format.\td{This is actually done in all examples.}




\iffalse
	-	Only executable definitions when evaluating = -1 (text + query)
	- Uniqueness of operation = -3
	- Anonymous operation = -2
	- Suscription = -5 (single root field)
	- Fields = 2 + 2 + 2 - 1 (introspection on union)
	- Field merging = 3 + 1 - 1 (variable) + 2 - 2  (variables) + 2  + 1
	- Leaf fields = 2 + 3 - 1 (type ext)
	- Argument names = 2 - 1 (directives) + 1 - 1 (directives) + 2  - 1 (type extension)
	- Required args = -5 (required arguments - non-null + default values)
	- Fragment uniqueness = -2 (fragments)
	- Fragment spread type existence = 1 - 3 (2 fragments y 1 directive) + 1 - 2 (fragments)
	- Fragments on comp types = 3 - 3 (fragments) + 2 - 2 (fragments)
	- Fragments must be used = -1
	- Fragments spread = -4 (fragments) + 2 - 2 (fragments) + 1 - 2 (fragments) + 1 - 2 (fragments) + 2 - 2  + 2 - 2 + 1 - 2 + 1 - 2 
	- Valid Input values = -5
	- Input objects = -3
	- Directives = -3
	- Variables = - 2 - 8 - 7 - 4 - 7
	
	Hechos : 2 + 2 + 2 + 3 + 1 + 2 + 2 + 2 + 3 + 2 + 1 + 2  + 1 + 1 + 3 + 2 + 2 + 1 + 1 + 2 + 2 + 1 + 1 = 41
	No hechos : -1 - 3 - 2 - 5 - 1 - 1 - 2 - 1 - 1 - 1 - 1 - 5 - 2 - 3 - 2 - 3 - 2  - 1 - 2 - 2 - 2 - 2 - 2 - 2 -2 - 2 - 5 - 3 - 3 - 2 - 8 - 7 - 4 - 7  = 92
	- Variables = -1 - 2 - 2 - 8 - 7 - 4 - 7 = 31
	- Fragments = -2 - 2 - 2 - 3 - 2 - 1 - 4 - 2 - 2 - 2 - 2 - 2 - 1 - 2 = 29 
	- Operations = -1 - 3 - 2 - 5 = 11
	- Directives = -1 -1 - 1 - 3 = 6
	- Valid input values = 5
\fi 
	
Finally, we also partially\et{is it only the error stuff that's missing?}\td{Same as above, it is missing not implemented features (errors, fragments, etc.)} implemented the Star Wars example defined in the reference implementation of \gql\footnote{https://github.com/graphql/graphql-js/tree/master/src/\_\_tests\_\_}.
We include the schema definition, except for the \texttt{secretBackstory} field which resolves to an error\td{Bc we are not handling errors}. The data modeled as a graph and 
evaluation of 9 queries (out of a total of 17), comparing them to the expected outputs. Of the not implemented cases, 3 correspond to errors, 2 to introspection, 1 to fragments, 2 to variables and 
one where execution is over an object with named properties\td{Idk, something more specific to JS than \gql}.
\td{There are 7 other about validation, which I didn't notice, but I can add them}
\td{In that same repo (https://github.com/graphql/graphql-js/tree/master/src) each subfolder (type, execution, etc) contain other \_\_tests\_\_ folders, but
I didn't delve too much into them -- some are a bit too particular to JS. They can still be adapted, but not now}
		

% Most of the development time was spent in the definition and proofs of normalization.
% We initially worked on the semantics as specified by \cite{gqlph}

%\et{have a brief section on limitations that lists everything from the spec that is not covered currently in \gcoql (operations beyond query, errors, etc)}

