% !TEX root = ./main.tex
\section{Query Transformation: Normalization}\label{sec:norm}

As a case study for query transformation, we tackle the normalization process proposed by \HP{}. This process is fundamental for the complexity results they prove. These results are based on two premises, the first one being that every query can be normalized and that the resulting query is semantically equivalent. Secondly, it is possible to define a new simplified evaluation function that is equivalent to the original semantics, if the queries are in normal form. For the former they provide a set of equivalence rules, which serve as rewriting rules, but do not provide proofs of correctness. Similarly, for the latter they do not prove the equivalence. In this section, we therefore describe our approach to tackling these issues. To begin with, we define the property of being in \textit{normal form} and the normalization procedure, as well as providing proofs of correctness and semantic equivalence. Finally, we define a new simplified semantics and prove the equivalence to the semantics defined in Section~\ref{subsec:semantics}.

It is worth mentioning that most of our formalization effort was devoted to defining and establishing the correctness of this normalization procedure. In terms of code, definitions are coded in approximately $350$ lines, while proofs encompass around $1,200$ lines. cThe complete definitions and proofs can be found in the files \texttt{QueryNormalForm.v} and \texttt{QueryNormalFormLemmas.v}.

\subsection{Normal form}

The notion of \textit{normal form} is defined by the conjunction of two properties; being \textit{grounded} and being \textit{non-redundant}.
\HP{} refers to the former as being in \textit{ground-typed normal form}. We believe this naming is confusing so we decide to rename it to the previously described property.

Similarly to what is described in Section~\ref{subsec:query}, the normalization process requires information on the type in context where the queries might be defined. This is crucial as it guides the process on how queries are transformed.

% Throuhgout our development, we noticed that this definition given by \cite{gqlph} was too general when proving correctness of our normalization procedure. In particular, the definition states that the subqueries of a field selection can be either fields or fragments. This means that if there are two field selections with the same response name, one may have subqueries consisting of fields, while the other contains only inline fragments. This would cause issues when trying to remove redundancies in queries because one could not directly establish if the resulting queries satisfied the property.
\subsubsection*{Groundness}

Informally, the \textit{groundness} property refers to whether queries are completely specified down to the object types.  Consider the query depicted in Figure~\ref{fig:grounded} that requests the name of an animal. This query is not grounded because the field \texttt{name} is made over the \texttt{Animal} type, which is an abstract type. Now, consider the query to the right that is requesting the same information as the previous. In this case, the query is grounded since it completely specifies the object types to which it will request the information.

\begin{figure}[h]
\centering
\begin{subfigure}{.25\textwidth}
\begin{minted}[escapeinside=||, mathescape=true]{js}
query {
  goodboi {
    name
  }
}
\end{minted}
\end{subfigure}%
\begin{subfigure}{.25\textwidth}
\begin{minted}[escapeinside=||, mathescape=true]{js}
query {
  goodboi {
    |$\ldots$| on Dog {
	  name
    }
    |$\ldots$| on Pig {
      name
    }
  }	
}
\end{minted} 
\end{subfigure}

\caption{Not grounded query (left) and semantically equivalent grounded query (right).}
\label{fig:grounded}
\end{figure}

The main idea is that if a query is performed over an object type then the query should only be composed of field selections. In contrast, if the query is over an abstract type, then it should only be composed of inline fragments that specify the selections down to the object subtypes. In the former case, it does not make sense to use fragments to further specify a query because it is not possible to be more specific when querying an object. Meanwhile, in the latter case the query should clearly state what is being requested from each concrete subtype.

\begin{definition}
A GraphQL query $\varphi$ is \textit{grounded} if it satisfies the following conditions, where \texttt{ty} is the type in scope.
\begin{itemize}
	\item If \texttt{ty} is an object type, then $\varphi$ contains only fields.
	\item If \texttt{ty} is an abstract type (interface or union), then $\varphi$ contains only inline fragments. The type condition on these fragments must be Object types.
	\item Subqueries of $\varphi$ are \textit{grounded} w.r.t. to the field's return type or the fragments type condition.
\end{itemize}
\end{definition}

Our definition of groundness is slightly different from the one in \HP{}, nevertheless, we prove that our definition still implies being in ground-typed normal form.

\begin{minted}{coq}
Lemma are_grounded_in_ground_typed_nf (s : wfGraphQLSchema)
                                      (type_in_scope : Name)
                                      (queries : seq Query) :
        are_grounded s ty queries ->
        are_in_ground_typed_nf s queries.
\end{minted}

\subsubsection*{Non-redundancy}

Informally, the notion of non-redundancy refers to whether there are no queries that may produce repeated results. For example, the two queries below depict the cases of a redundant and a non-redundant query. The first one is requesting the field \texttt{name} twice to the same type and uses two fragments with the same type condition. If no collection and merger of fields is performed during the evaluation, this would produce two values with key \texttt{name}. Meanwhile, the second case requests information about each type only once. It is important to notice that, even though it requests the \texttt{name} field in both,  this is considered not redundant. The reason is that only one fragment will actually be executed at a time, depending on the concrete object value that is used to evaluate the query, therefore not repeating the evaluation of the field.
\td{Should this be another figure?}
\begin{minipage}[t]{.25\textwidth}
\begin{minted}[escapeinside=||, mathescape=true]{js}
// Redundant query
query {
    goodboi {
        name
	name
	|$\ldots$| on Dog {
	    name
	}
	|$\ldots$| on Dog {
	    friends { |$\ldots$| }
	}
    }
}
\end{minted}
\end{minipage}%
\begin{minipage}[t]{.25\textwidth}
\begin{minted}[escapeinside=||, mathescape=true]{js}
// Non-redundant query
query {
    goodboi {
        |$\ldots$| on Dog {
	    name
	}
	|$\ldots$| on Pig {
	    name
	}
    }	
}
\end{minted} 
\end{minipage}

\begin{definition}
A GraphQL query $\varphi$ is \textit{non-redundant} if it satisfies the following conditions.
\begin{itemize}
    \item There is at most one field selection with a given response name, for a particular depth of the query tree.

    \item There is at most one inline fragment with a given type condition, for a particular depth of the query tree.
    
    \item Subqueries are non-redundant.
\end{itemize}
\end{definition}

Our definition of non-redundancy assumes that the queries are grounded, much like \HP{}, mainly to simplify the implementation. The difficulty arises from using inline fragments and comparing their contents appropriately. This difficulty is further enhanced by the fact that the \spec{} currently allows using fragments that can possibly span over several unrelated types\footnote{https://graphql.github.io/graphql-spec/June2018/\#sec-Fragment-spread-is-possible}\footnote{https://github.com/graphql/graphql-spec/issues/367}\footnote{Example of inline fragments spanning to unrelated types - https://tinyurl.com/y4uxz3gu}\td{Should we expand on this somewhere?}.

\subsection{Normalization procedure}\label{subsec:normalization}

The normalization procedure can be understood as a form of abstract interpretation, which evaluates queries using only static information about the type in context. Figure~\ref{fig:normalize} displays pseudocode describing the process when the head of the list of queries is a field selection with subqueries. We do not include the complete definition because of space limitations but the complete definition can be found in the file \texttt{QueryNormalForm.v}.


\begin{figure*}[h]
\centering
\begin{tabular}{c}
\begin{lstlisting}[
mathescape=true,
style=code]
(*@\textbf{BEGIN}@*) normalize
 (*@\textbf{INPUT}@*) schema, type_in_context, queries
 CASE queries WITH
  | nil => nil
  | CONS (response_name [args] { subqueries }) queries => 
    LET return_type := LOOKUP (schema, type_in_scope, response_name) IN
    LET collected := COLLECT (schema, response_name, type_in_scope, queries) IN
    LET merged := CONCAT (subqueries,  MERGE collected) IN
    LET filtered := FILTER (response_name, queries) IN
    
    IF IS_OBJECT_TYPE (schema, return_type) THEN 
     CONS
      response_name [args] { normalize (schema, return_type, merged)} 
      normalize (schema, type_in_scope, filtered)
    ELSE 
     LET subtypes := SUBTYPES (schema, type_in_scope) IN
     CONS
      response_name [args] { MAP ($\lambda$ subtype => INLINE (subtype, normalize (schema, subtype, merged))) subtypes } 
      normalize (schema, type_in_scope, filtered)
     
  | ...
(*@\textbf{END}@*) normalize
\end{lstlisting}
\end{tabular}
\caption{Pseudocode for the normalization procedure, showcasing field selections with subqueries.\td{Should we add brief defs. of COLLECT, MERGE, etc.?}}
\label{fig:normalize}
\end{figure*}

The complete normalization process is actually composed of two separate functions; \texttt{normalize} that performs all the actual work but under the assumption that the type in context is an object type, and \texttt{normalize\_queries} that makes no assumption but only pipes the work to the former. The main process can be informally described as consisting of two subprocesses that deal with the two aforementioned properties.
\iffalse
\begin{itemize}
    \item Grounding: Selections are either wrapped with inline fragments or lifted from an inline fragment.

    \item Merging: Fields with the same response name have their subqueries merged into a single selection.
    
    %Whenever a field is encountered, the procedure tries to find all fields with the same response name and merge their subqueries. It then proceeds to remove them from the list to ensure \textit{non-redundancy}. Comparing it to the the semantics, this is equivalent to the case when we evaluate a field and collect similar ones.
    
    %Since it is assumed that the type in context is an Object type, it will try to transform the query such that there are only fields left. This means it will try to get rid of inline fragments and lift their subqueries as much as possible. Much like if we were standing on a node in the graph, we only evaluate fragments and subqueries that make sense for that node's type (which is an Object type). In the case of fields, it will first check on its return type. If it is an abstract type, then it will create a cover of all possible concrete subtypes of the abstract type, by wrapping the subqueries with inline fragments. Otherwise, it will proceed recursively. Once again, this is like finding the neighbors of a node. Since a priori it doesn't know the neighboring nodes that may be encountered, the procedure anticipates all possible scenarios.
\end{itemize}
\fi

The first subprocess tackles the groundness of queries, and corresponds to the \texttt{if-else} block as well as the mapping in line 19 of Figure~\ref{fig:normalize}. The grounding is done by either wrapping selections with inline fragments, whenever the type in context is an abstract type, or by lifting nested selections from inside fragments, whenever their type conditions are compatible with the object type in context\td{This is related to the fact that GraphQL allows invalid fragments}. This process can be illustrated with the example for Figure~\ref{fig:grounded}. Once again, the query to the left is not grounded since it requests the name of an animal without being specific down to the object types. The normalization process will then produce the query to the right, by wrapping the selection using fragments with the subtypes of the \texttt{Animal} type, namely \texttt{Dog} and \texttt{Pig}.

\iffalse
\td{This is the same example as above, so it can be reused}
\begin{minted}[escapeinside=||, mathescape=true]{js}
// Not grounded query
query {
    goodboi {
        name
    }
}
// Normalized query
query {
    goodboi {
        |$\ldots$| on Dog {
	    name
	}
	|$\ldots$| on Pig {
	    name
	}
    }	
}
\end{minted} 
\fi


\td{Unnecessary?} Next, to tackle inline fragments that are unnecessary or that specialize selections in the context of an object type, let us consider the following queries. The first query includes two fragments that are not necessary; a fragment with type condition \texttt{Query} and one with type condition \texttt{Animal}. The procedure eliminates both fragments, by lifting the subqueries.

\begin{minted}[escapeinside=||, mathescape=true]{js}
// Not grounded query
query {
    |$\ldots$| on Query {
        goodboi {
            |$\ldots$| on Dog {
	        |$\ldots$| on Animal {
		    name
		}
	    }
	}
}
// Normalized query
query {
    goodboi {
        |$\ldots$| on Dog {
	    name
	}
    }	
}
\end{minted} 

When it comes to removing redundancies in a query, there is a second subprocess that handles it, which roughly correspond to lines 7-9 in Figure~\ref{fig:normalize}. The process collects fields that share the same response name, merging their subqueries into a single selection. For example, the first query below is redundant since it requests the same \texttt{goodboi} field twice, and the subqueries in both cases also contain repeated \texttt{name} selections. The normalization process then merges the selections with the same response name, leaving only one occurrence of each case.

\begin{minipage}[t]{.25\textwidth}
\begin{minted}[escapeinside=||, mathescape=true]{js}
// Redundant query
query {
    goodboi {
        name
    }
    goodboi { 
 	name
    } 
}
\end{minted}
\end{minipage}%
\begin{minipage}[t]{.25\textwidth}
\begin{minted}[escapeinside=||, mathescape=true]{js}

// Normalized query
query {
    goodboi {
        |$\ldots$| on Dog {
	    name
	}
	|$\ldots$| on Pig {
	    name
	}
    }	
}
\end{minted} 
\end{minipage}

% With this definition, we define a second one, which makes no assumption on the type in context. This procedure only checks what kind of type it receives and either pipes the job to the previous one, or covers the queries with the possible concrete subtypes (and then pipes the work to the previous definition).

\iffalse
\begin{minted}{coq}
 Definition normalize_queries (s : wfGraphQLSchema)
                             (type_in_scope : Name)
                             (queries : seq Query) :
                                         seq Query :=
    if is_object_type s type_in_scope then
        normalize s type_in_scope queries
    else
        [seq on t { normalize s t queries } |
            t <- get_possible_types s type_in_scope].

\end{minted}
\fi

As a final note, it is worth mentioning that the subprocesses mentioned are not defined as separate functions, but occur interleaved in the normalization function. As a consequence, the definition is highly\td{?} non-structural but can be easily expressed using the Equations library. The  similarity between the normalization function and the semantics also eases reasoning about the preservation of the semantics. \td{Mention something about how we first split the definition into these 2 subprocesses but ended up being harder to reason about?}

We now move onto proving correctness of the normalization procedure and the preservation of the semantics.


% Finally, with these properties and definitions we prove the premises proposed by \HP{}. We leave that discussion, about \HP{}'s approach and ours, to Section~\ref{subsec:discussion}.

\subsection{Proofs of correctness and preservation}

As described initially in this paper, \HP{} base the complexity results over GraphQL queries on two premises; queries can be normalized, preserving their semantics, and there is an equivalent simplified function to evaluate queries in normal form. We tackle the first premise by establishing correctness of our normalization procedure and then preservation of semantics.

First, we prove that the procedure is correct and returns queries in normal form, by proving separately that the resulting queries are grounded and non-redundant\footnote{Since we prove that grounded implies ground-typed normal form, we can also prove ground-typed normal form for the normalization function.}. Both are proved by well-founded induction over the queries size\footnote{The notion of size includes the length of the list as well as the depth of the query tree.} and then mostly auxiliary lemmas over subtypes.  We consider the case where the type in scope is an Object type and the general case. 

\begin{minted}[escapeinside=||,mathescape=true]{coq}
Lemma normalize_are_grounded (s : wfGraphQLSchema)
                             (ty : Name)
                             (|$\varphi$| : seq Query) :
    is_object_type s ty ->
    are_grounded s ty (normalize s ty |$\varphi$|).

Lemma normalize_are_non_redundant (s : wfGraphQLSchema)
                                  (ty : Name)
                                  (|$\varphi$| : seq Query) :
    is_object_type s ty ->
    are_non_redundant (normalize s ty |$\varphi$|).

\end{minted}

Next, we prove that the semantics are preserved for the resulting queries. To begin with, we prove the case where the type in context is the same as the type of the node where queries are being normalized. Lifting this to top level, it corresponds to normalizing over the Query type and evaluating on the root node (whose type is the same, due to graph conformance). We then extend this notion to normalization over any type in context, \texttt{ty}, but with the restriction that the node's type must be subtype of \texttt{ty}. Once again, this is valid at top level over the Query type and the root node. Conformance of the graph also ensures that normalization and evaluation over neighboring nodes is preserved. The proof is also performed by well-founded induction over the queries size and mostly auxiliary lemmas about graph conformance.

\begin{minted}[escapeinside=||,mathescape=true]{coq}
Lemma normalize_exec (s : wfGraphQLSchema)
                     (g : conformedGraph s)
                     (u : node)
                     (|$\varphi$| : seq Query) :
    u |\textbackslash|in g.(nodes) ->
    s, g |$\vdash$| |$\llbracket$| normalize s u.(ntype) |$\varphi$| |$\rrbracket$| in u with coerce =
    s, g |$\vdash$| |$\llbracket$| |$\varphi$| |$\rrbracket$| in u with coerce.

Theorem normalize_queries_exec (s : wfGraphQLSchema)
                               (g : conformedGraph s)
                               (u : node)
                               (ty : Name)
                               (|$\varphi$| : seq Query) :
    u |\textbackslash|in g.(nodes) ->
    u.(ntype) |\textbackslash|in get_possible_types s ty ->
    s, g |$\vdash$| |$\llbracket$| normalize_queries s ty |$\varphi$| |$\rrbracket$| in u with coerce =
    s, g |$\vdash$| |$\llbracket$| |$\varphi$| |$\rrbracket$| in u with coerce.

\end{minted}

This concludes our proofs of normalization and establish that the first premise used by \HP{} is correct in the context of our system.
The next section continues with the second premise, namely the definition of a simplified version of the semantics
and the proof of equivalence.

\subsection{Simplified semantics}

As proposed by \HP{}, one of the main properties of queries in normal form is that they produce non-redundant responses\td{They actually say "unique responses" - they don't define non-redundancy for responses.}, without the need of any collection and merger of fields. This property allows defining a second evaluation function $\ll \varphi \gg_{G}$, similar to the one defined in Section~\ref{subsec:semantics} but without any filtering and collecting of fields.

We implement the simplified semantics following \HP{} and then prove the premise that, for queries in normal form, both $\llbracket \varphi \rrbracket_{G}$ and $\ll \varphi \gg_{G}$ produce the same response. We do not include the complete definition but they can be found in the files \texttt{QuerySemantics.v} and \texttt{QuerySemanticsLemmas.v}. The proof is once again performed by induction over the size of the queries.

\begin{minted}[escapeinside=||,mathescape=true]{coq}
Theorem exec_equivalence (s : wfGraphQLSchema)
                         (g : conformedGraph s)
                         (u : node)
                         (|$\varphi$| : seq Query) :
    are_in_ground_typed_nf s |$\varphi$| ->
    are_non_redundant |$\varphi$| ->
    s, g |$\vdash$| |$\llbracket$| |$\varphi$| |$\rrbracket$| in u with coerce =
    s, g |$\vdash$| |$\ll$| |$\varphi$| |$\gg$| in u with coerce.
\end{minted}

This result concludes the normalization process. We have described our approach to proving both premises exploited by \HP{}. As mentioned earlier, these are fundamental for their complexity results over GraphQL queries but were not proven true. We believe our approach properly and rigorously addresses them. The following section discusses
