%!TEX root = ./main.tex
\section{Conclusions}
\label{sec:conclusion}

In this paper, we presented the first mechanized formalization of \gql's specification, implemented in the \coq proof assistant.
We formalize a considerable portion\td{Missing things} of the schema definition DSL, queries and the validity of both, and their semantics over a graph data model. 
With this formalization, we then study the query transformation proposed and exploited by \HP. 
We provide an algorithmic definition of the normalization process, proving it correct and semantics preserving. after 
addressing some minor issues in the original definitions.

While this serves as a foundational step towards fully formalizing \gql, there are several lines of future development.
\paragraph{Extending \gcoql} The current implementation does not include some features such as mutation, directives and non-null types. 
By following a similar approach as in~\cite{empiricalgql}, one could identify key features to be included in future releases.

\paragraph{Automation} Extending \gcoql to include new features should be fairly simple. In particular, this means that existent proofs should 
not be easily broken or they should be easily fixable. In order to do so, it is necessary to modularize and improve the current state of tactics used in
our proofs.

\paragraph{Extraction \& Testing} While the current implementation attempts to be as close as possible to the grammar and algorithmic descriptions
given in the \spec, it can be improved by extracting it to other languages and comparing it directly to the reference implementation, following the examples of~\cite{jscert, coqr}.
We believe that the current design decisions allow for a smooth extraction.

\paragraph{Data model \& Spec semantics} A new evaluation function can be implemented, such that it is closer to the \spec's and defined over any data model. 
The current semantics over graphs can then be a particular instance of this evaluation function, after addressing the limitations of the data model (\S\ref{subsec:discussion}). 
We believe that the main difficulty lies in faithfully representing nested list types in a graph.
\td{As we once talked, the spec's semantics can be defined using functors. Modularizing the current semantics 
should be quite simple tbh -- I haven't done so bc: 
1) I'm not entirely sure how to model functors in Coq lol (I have seen implementations, but I haven't had the time to properly use them, understand them, etc),
2) Still the issue with the nested list types in graphs}



\td{Collab with GraphQL foundation/community?}

